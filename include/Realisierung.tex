\chapter{Realisierung}
% TODO Beschreibung der Umsetzung der definierten Ziele, einschliesslich der aufgetretenen Schwierigkeiten und Einschränkungen

% Dies ist das Hauptkapitel Ihrer Arbeit! Hier wird die Umsetzung der eigenen Ideen und Konzepte
% (Kapitel 3) anhand der gewählten Methoden (Kapitel 4) beschrieben, inkl. der dabei aufgetretenen
% Schwierigkeiten und Einschränkungen.

\section{Projektmanagement}

\subsection{Anforderungen}
\label{sec:Anforderungen}

Die Anforderungen an das Projekt haben sich aus der Aufgabenstellung (siehe Appendix~\ref{ch:assignment}) sowie aus dem Requirement Engineering (siehe Dazu den Abschnitt~\ref{sub:RequirementsEngineering}).

Die folgende Tabelle~\ref{tab:requirements} zeigt alle Anforderungen auf.
Die erste Spalte ''Id'' gibt jeder Anforderung einen eindeutigen Kennzeichner, um diese einfacher zu Referenzieren.
Die Anforderung selber ist in der Spalte ''Anforderung'' genauer beschrieben.
Die Spalte ''Art'' gibt an, ob es sich bei der Anforderung um eine funktionale oder nicht-funktionale Anforderung handelt.
Nicht jede Anforderung muss zwingend erfüllt werden. Deshalb kennzeichnet die Spalte ''Optional'', ob die jeweilige Anforderung zwingend erfüllt werden muss.

\newcommand*{\reqref}[1]{
    \hyperref[{req:#1}]{\ref{req:#1}-#1}
}
\newcommand*{\seereq}[1]{(siehe Anforderung \reqref{#1})}
\newcommand*{\rid}[1]{\label{req:#1}-#1}
\begin{longtable}{N p{8.5cm} l l}
    \toprule
    \multicolumn{1}{r}{\bfseries Id} & \bfseries Anforderung                                                                                                                                                           & \bfseries Art & \bfseries Optional \\ \midrule
    \endhead
    \rid{SDTF}  & Der Stand der Technik/Forschung muss ermittelt werden und in den Bericht eingebunden werden.
                & Nicht Funktional & Nein  \\ \midrule


    \rid{TINF}  & Es ein Teststand muss aufgebaut werden welches es erlaubt verschiedene Performance-Messungen
                  an einem Netzwerk bestehend mehreren i2pd-Knoten. & Funktional & Nein \\ \midrule
    \rid{TKON}  & Es soll ein Konzept erstellt werden für den Teststand, wie dieser aussehen soll, wie er funktioniert und was genau gemessen werden soll. & Nicht-Funktional & Ja \\ \midrule
    \rid{TINF}  & Die am Teststand durchgeführten Messungen sollten so gut wie möglich reproduzierbar sein. Die gleiche Messung sollte soweit wie möglich dieselben Resultate liefern. & Nicht-Funktional & Nein \\ \midrule
    \rid{TCNF}  & Der Teststand muss konfigurierbar sein (inwiefern?), damit verschiedene Messungen durchgeführt werden können.  & Funktional & Nein \\ \midrule
    \rid{TSCL}  & Das Testnetzwerk soll zwischen 8 Knoten bis maximal 256 i2pd-Knoten unterstützen. Dies muss konfigurierbar sein, dass Testnetzwerk sollte skalieren können. & Funktional & Nein \\ \midrule
    \rid{TISO}  & Das Testnetzwerk soll isoliert sein vom realen I2P-Netzwerk. & Nicht-Funktional & Nein \\ \midrule
    \rid{TLAT}  & Die Latenz von Nachrichten die über das I2P-Testnetzwerk gesendet werden, soll gemessen werden. & Funktional & Nein \\ \midrule
    \rid{TLIM}  & Es muss möglich sein im Teststand die verfügbare Bandbreite von einzelnen i2pd-Knoten einzustellen. & Funktional & Nein \\ \midrule
    \rid{TVRS}  & Man soll schnell (wie schnell?) ein neues Experiment mit neuen Einstellungen starten können. & Nicht-Funktional & Nein \\ \midrule
    \rid{TPER}  & Man geht davon aus das die Messungen lange dauern könnten, da jeweils ein ganzes Netzwerk und dessen Traffic erstellt und gehandhabt werden muss. Es macht Sinn die Ausführungszeiten von Messungen kurz zu halten, damit mehr Messungen durchgeführt werden können. & Nicht-Funktional & Ja \\ \midrule
    \rid{DOCS}  & Am Ende des Projekts muss dieser Bachelorarbeit abgegeben werden. Dieser Bericht soll das Projekt, die verwendeten Methoden, Evaluationsprozesse, die Umsetzung und Resultate beschreiben.
                & Nicht Funktional & Nein \\ \midrule
    \rid{ITER}  & Ein iteratives Projektmanagement wird bevorzugt für kurze Feedback-Zyklen. Da ich alleine am Projekt arbeite sollte so auch weniger Overhead entstehen. & Nicht Funktional & Ja \\ \midrule
    \rid{PRES}  & Es muss eine Zwischenpräsentation gehalten werden, welche das Projekt und das weitere Vorgehen erklärt.
                & Nicht Funktional & Nein \\ \midrule
    \rid{WEBA}  & Es muss ein Web-Abstract erstellt werden, der das Projekt kurz zusammenfasst. Der Web-Abstract wird veröffentlicht und muss eine Woche vor Projektende vom Betreuer abgenommen werden.
                & Nicht Funktional & Nein \\ \midrule
    \rid{PVID}  & Es muss ein 90 Sekunden Video erstellt werden, welches diese Bachelorarbeit kurz erklärt. Das Video wird am Start der Schlusspräsentation abgespielt.
                & Nicht Funktional & Nein \\ \midrule
    \bottomrule
    \caption{Requirements}\label{tab:requirements}
\end{longtable}
