\chapter{Realisierung}
% TODO Beschreibung der Umsetzung der definierten Ziele, einschliesslich der aufgetretenen Schwierigkeiten und Einschränkungen

% Dies ist das Hauptkapitel Ihrer Arbeit! Hier wird die Umsetzung der eigenen Ideen und Konzepte
% (Kapitel 3) anhand der gewählten Methoden (Kapitel 4) beschrieben, inkl. der dabei aufgetretenen
% Schwierigkeiten und Einschränkungen.

\section{Projektmanagement}

\subsection{Resultate}

Die folgende Tabelle~\fullref{tab:resultate} beschreibt alle Resultate und Unterresultate die während dieser Bachelorarbeit erstellt werden.
Diese Tabelle wurde zusammengestellt anhand der Anforderungen im Abschnitt~\ref{sec:Anforderungen}.

Jedes Resultat hat einen eindeutigen Identifier in  der Spalte ''Nr'' aufgelistet.
Die Spalte ``Anforderung'' bezieht sich darauf, welche Anforderungen für das jeweilige Resultat massgebend sind.
In der letzten Spalte ''Geschätzter Arbeitsaufwand'', wurde zu beginn des Projekts abgeschätzt wie viel Arbeitsaufwand (in Stunden) das Resultat verursacht.

\begin{longtable}{p{0.8cm} l p{3.5cm} p{2cm}}
    \toprule
    \bfseries Nr & \bfseries Resultat & \bfseries Anforderung& \multicolumn{1}{p{3cm}}{\bfseries Geschätzter Arbeitsaufwand} \\
    \midrule \endhead
    D            & \textbf{Dokumentation}                                       & \reqref{DOCS} & \textbf{Total 44h}  \\
    \midrule                                                               
    D.1          & \; Dokumenten Layout                                &       & 8h   \\
    D.2          & \; Aufbau des Berichts                              &       & 4h   \\
    D.3          & \; Titelseite                                       &       & 2h   \\
    D.4          & \; Zusammenfassung / Abstract                       &       & 2h   \\
    D.5          & \; Einleitung                                       &       & 4h   \\
    D.6          & \; Beschreibung Motivation / Problem                &       & 2h   \\
    D.7          & \; Beschreibung der Aufgabenstellung                &       & 2h   \\
    D.8          & \; Beschreibung der Ziele / Vision                  &       & 1h   \\
    D.9          & \; Fragestellungen / Hypothesen                     &       & 2h   \\
    D.11         & \; Reflektion / Fazit                               &       & 3h   \\
    D.12         & \; Persönliches Projektfazit                        &       & 1h   \\
    D.13         & \; Ausblick                                         &       & 4h   \\
    D.14         & \; Anhang                                           &       & 3h   \\
    D.15         & \; Zwischenpräsentation                             & \reqref{PRES}  & 6h  \\
    \midrule                                                               
    R            & \textbf{Research State of the Art}                           & \reqref{SDTF} \reqref{DOCS}  & \textbf{Total 40h} \\
    \midrule                                                               
    R.1          & \; Literatur sammeln (Recherche)                    &       & 12h  \\
    R.2          & \; Bibliographie erstellen                          &       &  4h  \\
    R.3          & \; P2P Networks                                     &       &  2h  \\
    R.3.1        & \;   - Latenz / Bandbreite / Performanz             &       &  2h  \\
    R.4          & \; Beschreibung I2P                                 &       & 10h  \\
    R.4.1        & \; -- Begrifflichkeiten                              &       &  4h  \\
    R.4.2        & \; -- Funktionsweise                                 &       &  2h  \\
    R.4.3        & \; -- Bandbreite                                     &       &  2h  \\
    R.4.4        & \; -- Latenz                                         &       &  2h  \\
    R.5          & \; Deployment von Testnetzwerken                    &       &  4h  \\
    R.6          & \; Beschreibung der Wissenschaftlichen Methode      &       &  2h  \\
    R.7          & \; Metriken für die Auswertung                      & \reqref{TPER} \reqref{TISO} \reqref{TREP}  &  6h  \\
    \midrule                                                               
    K            & \textbf{Testkonzept          }                               & \reqref{TKON} \reqref{DOCS}  & \textbf{Total 46h}  \\
    \midrule                                                               
    K.1          & \; Beschreibung Ideen / Konzepte                    &       &  4h  \\
    K.2          & \; Anforderungen an den Teststand                   &       &  4h  \\
    K.3          & \; Teststrategie                                    &       &  4h  \\
    K.4          & \; Architektur Teststand                            &       &  4h  \\
    K.5          & \; Komponentendiagramm                              &       &  2h  \\
    K.6          & \; Beschreibung was gemessen werden soll            &       &  8h  \\
    K.7.1        & \; -- Bandbreite                                     & \reqref{TLIM} &  2h  \\
    K.7.1        & \; -- Anzahl Tunnels                                 & \reqref{TCNF}    &  2h  \\
    K.7.1        & \; -- Latenz von Nachrichten                         & \reqref{TLAT}    &  2h  \\
    K.7.1        & \; -- Ressourcenauslastung eines Knotens             & \reqref{TPER}    &  2h  \\
    K.8          & \; Beschreibung wie gemessen wird                   &       & 10h  \\
    K.8.1        & \; -- Isolation des Netzwerks                        & \reqref{ORDR}    &  2h  \\
    K.8.1        & \; -- Verschiedene Netzwerksegmente                  & \reqref{ORDR}    &  2h  \\
    K.8.2        & \; -- Latenz                                         & \reqref{ORDR}    &  2h  \\
    K.8.3        & \; -- Bandbreite                                     &       &  2h  \\
    K.8.4        & \; -- Konfigurationsmöglichkeiten                    & \reqref{TCNF} &  2h  \\
    K.9.5        & \; \glsname{ci}                                     & \reqref{TVRS} &  6h  \\
    K.9.6        & \; Beschreibung der Auswertungsmethode              &               &  4h  \\
    \midrule                                                               
    S            & \textbf{Teststand Design und Implementation}                 & \reqref{TINF} \reqref{DOCS} & \textbf{Total 124h} \\
    \midrule
    S.1          & \; Software Design                                  &       &  16h \\
    S.1          & \; Implementation                                   &       &  72h \\
    S.2.1        & \; -- Deployment des Testnetzwerkes                  & \reqref{TVRS} \reqref{TPER} &  8h \\
    S.2.2        & \; -- Netzwerksegmentierung                          & \reqref{TISO} &  8h \\
    S.2.3        & \; -- Konfigurationsmöglichkeiten                    & \reqref{TCNF} &  8h \\
    S.2.4        & \; -- Skalierung                                     & \reqref{TSCL} &  8h \\
    S.2.5        & \; -- Bandbreitenbeschränkung                        & \reqref{TLIM} &  8h \\
    S.2.6        & \; -- Reproduzierbarkeit                             & \reqref{TREP} &  8h \\
    S.2.7        & \; -- Latenzmessung                                  & \reqref{TLAT} &  8h \\
    S.2.8        & \; -- Messung der Ressourcenauslastung               & \reqref{TPER} &  8h \\
    S.2.9        & \; -- Verschiedene Testaufbauten                     & \reqref{TVRS} &  8h \\
    S.3          & \; Test des Labors                                  &       &  24h \\
    S.4          & \; Handbuch für den Teststand                       &       &  12h \\
    S.4.1        & \; -- Installation                                   &       &   2h \\
    S.4.3        & \; -- Konfiguration                                  &       &   2h \\
    S.4.3        & \; -- Ausführen von Messungen                        &       &   2h \\
    S.4.3        & \; -- Beschreibung der gesammelten Messdaten         &       &   4h \\
    S.4.3        & \; -- Beispiele                                      &       &   2h \\
    \midrule                                                                        
    A            & \textbf{Messung und Auswertung}                              & \reqref{EVAL} \reqref{DOCS}   & \textbf{Total 52h}  \\
    \midrule
    A.1          & \; Sammlung an Messdaten für die Auswertung         &        &  12h  \\
    A.2          & \; Beschreibung der Auswertungsmethode              &        &   4h  \\
    A.3          & \; Auswertung der Messungen                         &                    & 20h  \\
    A.3.1        & \; -- Einfluss der Knoten auf die Latenz             & \reqref{TLAT}      &  6h  \\
    A.3.2        & \; -- Einfluss der Anzahl Verbindungen auf die Latenz& \reqref{TLAT} \reqref{TLIM} &  6h  \\
    A.3.3        & \; -- Einfluss der Bandbreite auf Latenz             & \reqref{TLAT} \reqref{TLIM} &  6h  \\
    A.3.4        & \; -- Äussere Einflüsse / Unreinheiten               & \reqref{TREP} \reqref{TISO} &  2h  \\
    A.4          & \; Verschiedene Diagramme/Grafiken                  &       &  8h  \\
    A.5          & \; Auswertung der Anforderungen an den Teststand    & \reqref{TINF} &  4h  \\
    A.6          & \; Zusammenfassung der Auswertung                   &       &  4h  \\
    \midrule                                                               
    P            & \textbf{Project Management Dokumentation}                    & \reqref{DOCS} \reqref{ITER}  &  \textbf{Total 54h}  \\
    \midrule
    P.1          & \; Beschreibung Projektorganisation                 &       &  1h  \\
    P.3          & \; Projektmanagement Methode                        &       &  2h  \\
    P.2          & \; Beschreibung Projektumfang                       &       &  2h  \\
    P.4          & \; Projektplanung                                   &       &  8h  \\
    P.5          & \; Liste von Requirements                           &       &  4h  \\
    P.6          & \; Liste von Resultaten                             &       &  4h  \\
    P.9          & \; Arbeitsjournal                                   &       &  4h  \\
    P.10         & \; Meeting-Protokolle und Notizen                   &       & 29h  \\
    \midrule                                                               
                 & \bfseries  Geschätzter Arbeitsaufwand               & \textbf{Total} & \bfseries 360h \\
    \midrule
    \bottomrule
    \caption{Resultate}
    \label{tab:resultate}
\end{longtable}

Diese Liste von Resultaten ist auch ein guter Ausgangspunkt um davon Issues für den Issue-Tracker zu erstellen.
Jeder issue kann nun auch sortiert werden nach Resultat-Kategorien z.B. ''Recherche'', ''Projektmanagement'', ''Dokumentation'', ''Evaluation'', ''Präsentation'', etc.

\section{Systemarchitektur}

\section{Komponentendesign}

\section{Umsetzung / Programmierung}

\section{Testing}
