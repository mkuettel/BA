\makeglossaries
% \newacronym{RFID}{RFID}{Radio-Frequency Identification}
% \newglossaryentry{HF}{name={HF},description={High Frequency, RFID Tags im Frequenzbereich von 3-30MHz}}
\newglossaryentry{p2p}                 { name={Peer-to-Peer},                  text={Peer-to-Peer},           description={{
    Peer-to-Peer ist
    %TODO: define Peer-to-Peer
}}}
\newglossaryentry{refactoring}         { name={Refactoring},                   text={refactoring},            description={{
    ``The process of changing a software system in such a way that it does
        not alter the external behavior of the code yet improves its internal
        structure'' ---\cite{fowler_refactoring_2018}
}}}
\newglossaryentry{ci}                  { name={Continuous Integration},        text={continuous integration}, description={{
    ``Continuous Integration is a software development practice where members
    of a team integrate their work frequently,                                 usually each person integrates
    at least daily - leading to multiple integrations per day. Each integration
    is verified by an automated build (including test) to detect integration
    errors as quickly as possible.'' -- \cite{fowler_continuos_2014}
}}}
\newglossaryentry{pr}                  { name={Pull Request},                  text={pull request},           description={{
    Eine Funktion von Code-Hosting Plattformen und ein Prozess worin ein Entwickler einen andern anfragen kann ob er Änderungen in sein Software-Projekt aufnehmen will.
}}}
\newglossaryentry{saas}                { name={SaaS},                          text={Software as a Service},  description={{
            \glstext{saas} ist meist zentral gehostete Software welche Abonnemente für Ihre Kunden anbietet.
    For Example: Web-Mail, Dropbox
}}}
\newglossaryentry{dht}                 { name={DHT},        text={Distributed Hashed Table},                    description={{
    Eine \glstext{dht} ist eine auf verschiedenen Knoten oder Rechnern verteilte Key-Value Datenbank.
}}}
\newglossaryentry{e2e}                 { name={E2E},        text={End-to-End Encryption},                     description={{
    Von Endgerät zu Endgerät verschlüsselt. Das heisst ein Man-in-the-Middle Angreifer könnte nur verschlüsselten Traffic mitlesen.
}}}
\newglossaryentry{tor}                 { name={TOR},        text={The Onion Router},                          description={{
    Von Endgerät zu Endgerät verschlüsselt. Das heisst ein Man-in-the-Middle Angreifer könnte nur verschlüsselten Traffic mitlesen.
}}}
\newglossaryentry{i2p}                 { name={I2P},        text={The Invisible Internet Protocol},                          description={{
            Dezentrales nachrichtenorientiertes Mischnetzwerk indem anonym verschlüsselte Nachrichten ausgetauscht werden können \parencite[p.~1]{zantout_i2p_2011}. Die Abkürzung I2P wird auch als the Invisible Internet Project ausgelegt, um das gesamte Projekt nicht nur das Protokoll zu beschreiben.
}}}
\newglossaryentry{nixos-container}    { name={}, text={}, description={{{
    Containertechnologie ähnlich wie docker.
    Wichtig dabei ist, dass es sich hierbei nicht um einen Docker-Container, sondern um NixOS-Container handelt technologie handelt, welche auf systemd-nspawn Containern basiert..
    Im Hintergrund brauchen jedoch beide Container-Arten die gleichen Linux-Kernel-Features (namespaces, cgroups, ... ).
}}}
% TODO: add vm / container here?
