\makeglossaries
% \newacronym{RFID}{RFID}{Radio-Frequency Identification}
% \newglossaryentry{HF}{name={HF},description={High Frequency, RFID Tags im Frequenzbereich von 3-30MHz}}
\newglossaryentry{p2p}                 { name={Peer-to-Peer},                  text={Peer-to-Peer},           description={{
    Bei Peer-to-Peer-Netzwerk handelt es sich um eine verteile Architektur. Alle Peers sind dabei gleichberechtigte Teilnehmer. Peers oder Netzwerkteilnehmer stellen ein Teil ihrer Ressourcen dem Netzwerk zur Verfügung.
    %TODO: define Peer-to-Peer
}}}
\newglossaryentry{fullydistributed} { name={Vollständig verteilt},                   text={vollständig verteilt},            description={{
    Ein vollständig verteiltes Netzwerk zeichnet sich dadurch aus, das jeder Knoten
    im Netzwerk komplett selbstständig ist und seine eigene Daten verwaltet. Es gibt also im klassischen Sinne keine Server oder Clients.
}}}
\newglossaryentry{ci}                  { name={CI},        text={Continuous Integration}, description={{
    ``Continuous Integration is a software development practice where members
    of a team integrate their work frequently,                                 usually each person integrates
    at least daily - leading to multiple integrations per day. Each integration
    is verified by an automated build (including test) to detect integration
    errors as quickly as possible.'' -- \cite{fowler_continuos_2014}
}}}
\newglossaryentry{dht}                 { name={DHT},        text={Distributed Hashed Table},                    description={{
    Eine ``Distributed Hashed Table'' ist eine auf verschiedenen Knoten oder Rechnern verteilte Key-Value-Datenbank.
}}}
\newglossaryentry{tor}                 { name={TOR},        text={The Onion Router},                          description={{
    Softwarepaket für anonyme Kommunikation. Es verwendet ein globales Overlay-Netzwerk bestehend aus Relays, um die Benutzer vor Netzwerküberwachung 
}}}
\newglossaryentry{i2p}                 { name={I2P},        text={The Invisible Internet Protocol},                          description={{
            Dezentrales nachrichtenorientiertes Mischnetzwerk indem anonym verschlüsselte Nachrichten ausgetauscht werden können \parencite[p.~1]{zantout_i2p_2011}. Die Abkürzung I2P wird auch als the Invisible Internet Project ausgelegt, um das gesamte Projekt nicht nur das Protokoll zu beschreiben.
}}}
\newglossaryentry{iac}                { name={IaC}, text={Infrastructure as Code}, description={{
    Das Verwalten und Bereitstellen von Infrastruktur wie Netzwerke, Dienste physische und virtuelle Maschinen unter Verwendung von maschinenlesbaren Definitionen oder Programmcode.
    Somit kann die Infrastruktur automatisiert aufgebaut werden und Versionierungssysteme können einfach eingesetzt werden.
}}}
% \newglossaryentry{socks5}
% gossip protocol
