\chapter{Einleitung}
\label{ch:Einleitung}
% TODO Allgemeine Einleitung in die Arbeit

\section{Aufgabenstellung und Zielsetzung}

Siehe auch die komplette Aufgabenstellung im Anhang \fullref{ch:aufgabenstellung}

\section{Problem}

Man hat festgestellt das Aplikationen auf dem I2P-Netzwerk für Endbenuzter langsam erscheinen.
Damit Applikationen auf dem I2P-Netzkwerk gut funktioniern, soll ermittelt werden wie die Performance
verbesswert werden soll.

\section{Fragestellung}

Verringert sich die Latenz im I2P Netzwerk je mehr Knoten es gibt?
Verringert sich die Latenz im I2P Netzwerk je mehr Bandbreite es gibt?

\section{Vision}

Wird in dieser festgestellt, das mehr Knoten oder verschiedene Arten von Knoten, könnte man mit diese Erkenntnissen in i2p-Netzkwerk aufnehmen.
Diese Erkenntnis könnte auch unter umständen mehr Personen dazu bewegen selber i2p-Knoten zu betreiben.

% Welche Ziele, Fragestellungen werden mit dem Projekt verfolgt? Die Bedeutung, Auswirkung und
% Relevanz dieses Projektes für die unterschiedlichen Beteiligten soll aufgeführt werden.
% Typischerweise wird hier ein Verweis auf die Aufgabenstellung im Anhang gemacht.
