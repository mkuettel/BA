% !TEX root = BA-Bericht.tex
\chapter{Einleitung}
\label{ch:Einleitung}
% TODO Allgemeine Einleitung in die Arbeit

In der heutigen hochvernetzten und globalisierten Welt von Big-Data scheinen die Thematiken Datenschutz und Privatsphäre oft unterzugehen.

Das Internet Protokoll (IP) ist eines der Grundsteine für das heutige Internet.
Dieses wurde jedoch damals unter nicht den gesichtspunkten wie Datenschutz, Privatsphäre und Sicherheit designt.
Es ging wohl eher darum, dass überhaupt ein Protokoll da war, um Nachrichten in einem Netzwerk zu routen.

Mittlerweile gibt es jedoch verschiedene Anonymisierungsnetzwerke, welche in diesem Bereich abhilfe schaffen könnten.

Die Performance dieser Netzwerke ist jedoch aber nie so gut wie von IP direkt. (siehe Anonymitätstrilemma).
Für Benutzer lässt die Performance jedoch oft zu wünschen übrig.
Auch sonst gibt es für Benutzer nicht wirklich Argumente wieso, solch ein Netzwerk eingesetzt werden soll, da es kaum Anwendungen gibt.


\section{Aufgabe und Problemstellung}

Das freie Software- und Netzwerkprojekt DIVA.EXCHANGE\footnote{Webauftritt: \url{https://diva.exchange}} entwickelt einen Softwareprototypen DIVA.
Der Softwareprototyp soll aufzeigen, dass es möglich ist eine sichere und dezentrale Handelsplattform für digitale Werte zu erstellen, welche auch die Privatsphäre der Benutzer schützt.

Der Softwareprototyp besteht aus drei Schichten.
Einer Anonymisierungsschicht auf Netzwerkebene (\glsname{i2p}), einer Datenhaltungsschicht basierend auf einer Blockchain und eine darüberliegende Handels- und Verwaltungssoftware.


Es besteht die Annahme, dass die Performance des \glsname{i2p}-Netzwerk's nicht ausreichend ist, um für Endbenutzer schnell reagierende Applikationen auf dem Netzwerk anbieten zu können.
Kurz vor Beginn dieser Arbeit dauerte das Hin- und Zurücksenden einer Nachricht durch das \glsname{i2p}-Netzwerk (Roundtrip) etwa drei bis acht Sekunden.

Das Netzwerk ist verglichen mit dem \glsname{tor} klein und es können grössere Latenzen entstehen, um Anonymität für die Teilnehmer zu bieten:

\begin{itemize}
    \item Eine Nachricht wird über mehrere Hops versendet und wird mehrmals weitergeleitet bevor diese bei ihrem Ziel ankommt.
    \item Fallen Hops aus müssen die Nachrichten neu über eine andere Route übermittelt werden.
    \item Jede Nachricht ist mehrfach verschlüsselt und jeder Knoten muss jeweils eine Verschlüsselungsschicht beim weiterleiten einer Nachricht entfernen was viele CPU-Zyklen kostet.
\end{itemize}


Im Rahmen dieses Projekts soll folgende Problemstellung erarbeitet werden:

\begin{hyp}[H\ref{hyp:first}] \label{hyp:first}
    Ist eine steigende Anzahl von I2P-Knoten für die I2P-Netzwerk-Latenz (tiefer) vorteilhaft?
    Können Nachrichten schneller vom Sender zum Empfänger gelangen je mehr Knoten das I2P-Netzwerk hat?
\end{hyp}

\begin{hyp}[H\ref{hyp:second}] \label{hyp:second}
    Führt eine steigende Anzahl von I2P-Knoten für eine Vergrösserung der I2P-Netzwerk-Bandbreite?
\end{hyp}

\begin{hyp}[H\ref{hyp:third}] \label{hyp:third}
    Führt eine grössere Anzahl an Verbindungen zwischen den I2P-Knoten zu einer Verringerung der Latenz?
\end{hyp}

\begin{hyp}[H\ref{hyp:fourth}] \label{hyp:fourth}
    Führt eine grössere Anzahl an Verbindungen zwischen den I2P-Knoten zu einer Vergrösserung der I2P-Netzwerk-Bandbreite?
\end{hyp}

Damit Applikationen auf dem I2P-Netzwerk gut funktionieren, soll ermittelt werden wie die Performance
verbessert werden soll.

Siehe auch die komplette Aufgabenstellung im Anhang \fullref{ch:aufgabenstellung}.

\section{Vorgehen und Methode}


Es soll ein iteratives Projektmanagement verwendet werden
Zum überprüfen der Hypothese sollen Experimente/Sim

\section{Ziel}

Es gilt aufzuzeigen unter welchen Umständen und Rahmenbedingungen Anwendungen auf dem \glsname{i2p}-Netzwerk kürzere Latenzzeiten aufweisen
und somit für Endbenutzer schneller reagieren können. Das Niveau an Anonymität soll aber beibehalten bleiben.


\section{Vision}

Wird festgestellt, dass mehr Knoten in einem I2P-Netzwerk die Latenz verringert.
Diese Erkenntnis könnte auch unter Umständen mehr Personen dazu bewegen selber \glsname{i2p}-Knoten zu betreiben.
Dies wiederum würde das Netzwerk stärken und diverse Netzwerkeffekte könnten auftreten.
Zum Beispiel könnte dies dazu führen, das mehr Entwickler Applikationen für das Netzwerk erstellen und es auch aus Benutzerseite attraktiver wird.

% Welche Ziele, Fragestellungen werden mit dem Projekt verfolgt? Die Bedeutung, Auswirkung und
% Relevanz dieses Projektes für die unterschiedlichen Beteiligten soll aufgeführt werden.
% Typischerweise wird hier ein Verweis auf die Aufgabenstellung im Anhang gemacht.
