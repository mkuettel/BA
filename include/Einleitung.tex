\chapter{Einleitung}
\label{ch:Einleitung}
% TODO Allgemeine Einleitung in die Arbeit




\section{Aufgabenstellung}

Das freie Software- und Netzwerkprojekt DIVA.EXCHANGE\footnote{Webauftritt: \url{https://diva.exchange}} entwickelt einen Softwareprototypen DIVA.
Der Softwareprototyp soll aufzeigen, dass es möglich ist eine sichere und dezentrale Handelsplattform für digitale Werte zu erstellen, welche auch die Privatsphäre der Benutzer schützt.

Der Softwareprototyp besteht aus drei Schichten.
Einer Anonymisierungsschicht auf Netzwerkebene (\glsname{i2p}), einer Datenhaltungsschicht basierend auf einer Blockchain und eine darüberliegende Handels- und Verwaltungssoftware.

Siehe auch die komplette Aufgabenstellung im Anhang \fullref{ch:aufgabenstellung}.

\section{Problem}

Es besteht die Annahme, dass die Performance des \glsname{i2p}-Netzwerk's nicht ausreichend ist, um für Endbenutzer schnell reagierende Applikationen auf dem Netzwerk anbieten zu können.
Kurz vor Beginn dieser Arbeit dauerte das Hin- und Zurücksenden einer Nachricht durch das \glsname{i2p}-Netzwerk (Roundtrip) etwa drei bis acht Sekunden.

Das Netzwerk ist verglichen mit dem \glsname{tor} klein und es können grössere Latenzen entstehen, um Anonymität für die Teilnehmer zu bieten:

\begin{itemize}
    \item Eine Nachricht wird über mehrere Hops versendet und wird mehrmals weitergeleitet bevor diese bei ihrem Ziel ankommt.
    \item Fallen Hops aus müssen die Nachrichten neu über eine andere Route übermittelt werden.
    \item Jede Nachricht ist mehrfach verschlüsselt und jeder Knoten muss jeweils eine Verschlüsselungsschicht beim weiterleiten einer Nachricht entfernen was viele CPU-Zyklen kostet.
\end{itemize}


\section{Fragestellung}

Im Rahmen dieses Projekts soll folgende Problemstellung erarbeitet werden:

\begin{hyp}[H\ref{hyp:first}] \label{hyp:first}
    Ist eine steigende Anzahl von I2P-Knoten für die I2P-Netzwerk-Latenz (tiefer) vorteilhaft?
    Können Nachrichten schneller vom Sender zum Empfänger gelangen je mehr Knoten das I2P-Netzwerk hat?
\end{hyp}

\begin{hyp}[H\ref{hyp:second}] \label{hyp:second}
    Führt eine steigende Anzahl von I2P-Knoten für eine Vergrösserung der I2P-Netzwerk-Bandbreite?
\end{hyp}

\begin{hyp}[H\ref{hyp:third}] \label{hyp:third}
    Führt eine grössere Anzahl an Verbindungen zwischen den I2P-Knoten zu einer Verringerung der Latenz?
\end{hyp}

\begin{hyp}[H\ref{hyp:fourth}] \label{hyp:fourth}
    Führt eine grössere Anzahl an Verbindungen zwischen den I2P-Knoten zu einer Vergrösserung der I2P-Netzwerk-Bandbreite?
\end{hyp}

Damit Applikationen auf dem I2P-Netzkwerk gut funktioniern, soll ermittelt werden wie die Performance
verbesswert werden soll.

\section{Ziel}

Es gilt aufzuzeigen unter welchen Umständen und Rahmenbedingungen Anwendungen auf dem \glsname{i2p}-Netzwerk kürzere Latenzzeiten aufweisen
und somit für Endbenutzer schneller reagieren können. Das Niveau an Anonymität soll aber beibehalten bleiben.


\section{Vision}

Wird festgestellt, dass mehr Knoten oder verschiedene Arten von Knoten, könnte man mit diese Erkenntnissen in \glsname{i2p}-Netzkwerk aufnehmen.
Diese Erkenntnis könnte auch unter Umständen mehr Personen dazu bewegen selber \glsname{i2p}-Knoten zu betreiben.

% Welche Ziele, Fragestellungen werden mit dem Projekt verfolgt? Die Bedeutung, Auswirkung und
% Relevanz dieses Projektes für die unterschiedlichen Beteiligten soll aufgeführt werden.
% Typischerweise wird hier ein Verweis auf die Aufgabenstellung im Anhang gemacht.


\section{Struktur des Berichts}
% outline the order of information in the thesis

Im nächsten Kapitel~\fullref{ch:standdertechnik} ....

Im Kapitel~\fullref{ch:methode} wird beschrieben welche Methoden .... 

