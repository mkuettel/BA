\chapter{Stand der Technik}
\label{ch:StandDerTechnik}
% Auch "Stand der Forschung" oder "Stand der Praxis"

% Bezogen auf die eigenen Zielsetzungen und Fragestellungen soll aufgezeigt werden, wie andere
% dieses oder ähnliche Probleme gelöst haben. Worauf können Sie aufbauen, was müssen Sie neu
% angehen? Wodurch unterscheidet sich Ihre Lösung von anderen Lösungen? Für wissenschaftlich
% orientierte Arbeiten sei hier explizit auf (Balzert, S. 66 ff) verwiesen.

% TODO Historie des betroffenen Feld

\section{Technologische Grundlagen}
% TODO Wichtige technologischen Grundlagen / Wissenswertes

\subsection{P2P Netzwerke}

Scope: Nur I2P

\subsection{Performance / Bandwidth / Latency}

\subsection{I2P}

I2P ist ein \glsname{p2p}-Netzwerk.

\begin{itemize}
    \item Unabhängiges Netzwerk
    \item Low-Latency
    \item \glstext{e2e}
\end{itemize}

\cite{astolfi_i2p_2015}
\cite{astolfi_i2p_nodate}

\cite{timpanaro_birds_2012}

\cite{timpanaro_evaluation_2015}

A scaleable framework for anonymous communication
\cite{noauthor_i2p_nodate-8}


\cite{hoang_measuring_2019}
\cite{hoang_empirical_2018}

\cite{de_boer_invisible_2019}

\cite{zantout_i2p_2011}

Latency vs Tor Usability Bandwith and Latency Comparision
\cite{ehlert_i2p_2021}

\subsection{Wie I2P funktioniert}

\begin{itemize}
    \item Jeder Knoten leitet auch traffic für das Netzwerk weiter
    \item Jeder Knoten kann die länge 
\end{itemize}

\section{Technische Konzepte}
\label{sec:technischeKonzepte}
% TODO Konzepte in diesem Feld welche für den Leser relevant sind

\subsection{Tunnels}

Tunnels Configuration
\cite{noauthor_i2p_nodate-3}

\begin{itemize}
    \item Jeder Knoten kann die länge seiner Tunnel selber bestimmen. (Standardwert 3)
    \item Mehrere Verschlüsslungslayer je nach Tunnellänge (vgl. Onion-Routing)
\end{itemize}

\subsection{Garlic-Routing}

\begin{itemize}
    \item
\end{itemize}

\subsection{Netzwerk-Bootstrapping / Netdb}

Seeding etc.

Reseed Access
\cite{noauthor_i2p_nodate-7}

\section{Stand im Bezug auf eigenes Projekt}
% TODO Welche Forschung wurde in jüngster Zeit gemacht welche relevant für das eigene Projekt sind

Usability Inspection of Anonmity Networks
(\cite{abou-tair_usability_2009})

Usability Tetsts

(\cite{schomburg_anonymity_2009})

Evaluation of Anonymity Networks
(\cite{timpanaro_evaluation_2015})

\subsection{Performance}


Performance improvement using SSL IN I2P
\cite{vashi_performance_2015}

Performance I2P Webseite
\cite{noauthor_performance_nodate}

Performance History I2P Webseite
\cite{noauthor_performance_nodate-1}

Auf der Webseite von I2P auf der Seite ''Future Performance Improvements'' sind zudem zukünftige Performance Verbesserungsmöglichkeiten aufgelistet.
\cite{noauthor_future_nodate}
On I2P's website the page 

Improving I2P (2012)
\cite{timpanaro_improving_2012}


\subsection{I2P-Testnetzwerke}

\subsection{Metriken zur Messung}

Towards Measuring on the I2P Netzwork
\cite{wang_towards_2013}

\begin{itemize}
    \item Latenzmessung (abschicken/empfangen (Laport-Zeitstempel im schlimmsten fall))
    \item Messung der Bandbreite (ab welchem Layer)
    \item Ressourcenauslastung
\end{itemize}

\cite{timpanaro_monitoring_nodate}

\subsection{Testen im öffentlichen I2P Netzwerk}

\begin{itemize}
    \item family tag \cite{noauthor_family_nodate}
    \item verfälscht resultate
    \item deshalb isolieren
    \item öffentliche Metriken  I2P Metrics: \cite{noauthor_i2p_nodate-4}
    \item Netzwerk ist klein (deshalb braucht es so ein beweis wie hier)
\end{itemize}
