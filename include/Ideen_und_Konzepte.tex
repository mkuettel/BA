% !TEX root = BA-Bericht.tex
\chapter{Ideen und Konzepte}

Dieses Kapitel soll aufzeigen wie ein Teststand erstellt werden soll, um die Performance eines I2P-Testnetzwerkes unter verschiedenen Bedingungen und Umständen zu messen.

Es wird ein Konzept für einen Teststand vorgestellt, der den Anforderungen (siehe Abschnitt~\fullref{sub:Anforderungen}) gerecht werden soll.

Angefangen mit dem Abschnitt \ref{sec:grundidee} werden verschiedene Lösungsansätze für verschieden Aspekte vorgestellt.

\section{Grundidee}\label{sec:grundidee}
% TODO Beschreibung wie das Problem im Ansatz gelöst werden soll

% Hier geht es um die Fragestellung, wie Sie die formulierten Ziele der Arbeit erreichen wollen.
% Sie halten z.B. erste, grobe Ideen, skizzenhafte Lösungsansätze fest. Gibt es mehrere Wege, Ansätze
% um dieses Ziel zu erreichen, begründen Sie hier, warum Sie einen bestimmten Weg einschlagen.
% Beispiel für ein Softwareprojekt: Erste Gedanken über eine grobe Systemarchitektur. Ist z.B. eine
% Microservice-Architektur angebracht? Welche Alternativen bestehen, wo gibt es Problempunkte? Die
% Umsetzung, die Beurteilung der Machbarkeit und die detaillierte Beschreibung der umgesetzten
% Architektur sind dann Teil der Realisierung.

% Abgrenzung zu Kapitel 5 (Realisierung):
% - Besteht ein wesentliches Projektziel darin, für Ihre Kunden z.B. ein Security-Konzept, ein
% Kommunikations-Konzeptes, ein IT-Fachkonzept oder ein anderes Fach-Konzept zu erstellen, dann
% wird die Entwicklung dieser (fachlichen) Konzepte unter «Realisierung» beschrieben (sie sind ja der
% eigentliche Kern Ihrer Arbeit).
% - Besteht z.B. ein wesentliches Ziel der Arbeit darin, eine passende Software-Architektur zu
% evaluieren, dann gehören die entsprechenden Beschreibungen ins Kapitel 5.

Grundsätzlich soll ein Teststand aufgebaut werden an dem empirische Performance Messungen an einem I2P-Testnetzwerk durchgeführt werden können \seereq{TINF}.
Dabei gibt es verschiedene Herausforderungen und Aspekte die es zu beachten gilt.


\section{Reproduzierbarkeit}

Um ein Test auf einem solchen Testnetzwerk wiederholbar und reproduzierbar zu machen gibt es folgendes zu beachten \seereq{TREP}.

\begin{itemize}
    \item Das Testnetzwerk soll isoliert werden, um äussere Einflüsse durch das Netzwerk (auch das offizielle I2P-Netzwerk) zu vermeiden (siehe auch Abschnitt~\ref{sec:isolierung}).
    \item Infrastructure as Code: Ist die Testinfrastruktur als Programmcode beschrieben, kann dieser zu einem späteren Zeitpunkt wieder ausgeführt werden, um das Testnetzwerk wieder zu erstellen. Mehr dazu im Abschnitt~\ref{sec:deployment}.
    \item Reproduzierbarer Build von \lstinline|i2pd| und der Software auf einem Knoten
\end{itemize}


\section{Deployment}\label{sec:deployment}

Wichtig für das Deployment ist das schnell neue Tests gestartet und neue Testnetzwerke erstellt werden \seereq{TPER}.
Und dies mit verschiedenen Konfigurationen \seereq{TCNF}.
Die im Vorherigen Abschnitt bereits erwähnt soll das Deployment auch reproduzierbar sein \seereq{TREP}.
Es soll auch möglich sein verschieden grosse Testnetzwerke zu erstellen \seereq{TSCL}.


\subsection{Container oder Virtuelle Maschinen}

Um Ressourcen zu schonen aufgrund von kleinerem Overhead und Start-Up-Time von Containern im Gegensatz zu VMs sind diese wohl zu bevorzugen.
Container erlauben es schneller tests durchzuführen \seereq{TPER} aufgrund der Start-Up/Deployment-Time
aber auch Tests mit mehr Knoten zu machen, da weniger Ressourcen für einen einzelnen Knoten benötigt wird, da der Betriebssystemkernel geteilt wird.
Dies ist der Fall weil bei Containerlösungen (oder auch OS-Virtualisierung) im Gegensatz zu Virtuellen Maschinen der Betriebssystemkernel zwischen den Instanzen geteilt wird.

\subsection{Tools}

Es gibt verschiedene Tools für deployment von Netzwerken wie z.B. \lstinline|docker-compose|, \lstinline|kubernetes|, \lstinline|terraform|, oder auch \lstinline|nixops|.


\section{Konfigurierbarkeit}

Folgendes soll im Teststand konfigurierbar sein (\reqref{TCNF}):

\begin{itemize}
    \item Anzahl Knoten
    \item Maximale Bandbreite
    \item Hops (länge der Tunnel)
    \item In welchem Netzwerk (familiy netzwerk/öffentliches Netzwerk) (für Vergleich/optional)
\end{itemize}

Nix als konfigurationssprache würde es erlauben beliebige aspekte des Setups konfigurierbar zu machen.
Die meisten oben erwähnten Konfigurationsmöglichkeiten werden bereits durch verschiedene NixOS-Module (\lstinline|i2pd|, \lstinline|firewall|, ... ) angeboten.

\section{Latenzmessung}

\subsection{Latenzmessung einer Nachricht}

Um die Latenz oder Verzögerung einer Nachricht im Testnetzwerks zu messen \seereq{TLAT}, könnte man jeweils den Empfangs- sowie Sendezeitpunkt einer Nachricht speichern und voneinander subtrahieren.

\begin{equation}
    Latenz = Empfangszeitpunkt - Sendezeitpunkt
\end{equation}

Jedoch muss so sichergestellt werden, dass die Uhrzeiten zwischen allen Knoten synchron ist.


\section{Limitieren der Bandbreite}

\lstinline|i2pd|
erlaubt es die Bandbreite eines Knotens \seereq{TLIM} mittels der \lstinline|bandwidth| Konfigurationseinstellung \seereq{TCNF} zu limitieren.

\begin{itemize}
    \item Die Bandbreite eines i2pd Knoten kann limitiert werden in der Konfiguration mit der Einstellung bandwidth.  
    \item \cite{noauthor_i2p_nodate-3}
    \item Tc - Traffic control tool
    \item Nix Tooling
\end{itemize}

\section{Isolierung des Testnetzwerks}
\label{sec:isolierung}

Das Diagramm~\fullref{fig:i2p-testnetwork} zeigt den Aufbau des Testnetzwerks auf. Das Testnetzwerk soll standardmässig weder mit dem Internet verbunden noch mit dem System des Testers, um äussere Einflüsse zu verringern und durch isolation bessere Reproduzierbarkeit zu erreichen.

\begin{figure*}[ht]
  \includegraphics[width=1.0\textwidth]{i2p-testnetwork.png}
  \caption{I2P Testnetwork}\label{fig:i2p-testnetwork}
\end{figure*}

Bei der Firewall handelt es sich entweder um das Hostsystem (VM-Network) für das Testnetzwerk selber, welche die i2p-Knoten hostet, oder um einen extra Container der aller outgoing traffic kontrolliert.
Wichtig ist das der Tester, welcher das Netzwerk erstellt und die Tests durchführt sich auch ausserhalb des Testnetzwerkes befindet, um ungewollte Einflüsse zu vermeiden.

Die I2P-Knoten sind als Router dargestellt und befinden sich eigentlich alle in zwei Netzwerken gleichzeitig. Das normale IP Netzwerk und das I2P-Netzwerk.

Am Ende des Tests müssen die Testresultate an einem sicheren Ort ausserhalb des Testnetzwerks abgelegt werden, damit diese später analysiert werden können. (Resul tDB)

In der \lstinline|i2pd| Konfiguration gibt es zusätzlich das \lstinline|familiy| tag, welches die Knoten als zusammenhängend markieren würde, würden diese aus Versehen mit dem echten I2P Netzwerk reden. Dies ist auch interessant bezüglich Tests die am echten Netzwerk durchgeführt werden sollen.

%TODO: fullmesh?

\section{Bootstrapping und Verbindungsgrad der Knoten}

Ganz zu Beginn, wenn ein Knoten dem I2P Netzwerk beitritt muss dieser Wissen wo mindestens ein anderer Knoten ist.
Um die erste Liste von Knoten zu erhalten damit Knoten dem Netzwerk beitreten können, muss die liste von Knoten im Netzwerk zu beginn des Tests ''geseedet'' mit einer liste von Bekannten Knoten.

Ohne einen Reseed server ist dies möglich mit einem floodfill reseed:

\url{https://i2pd.readthedocs.io/en/latest/tutorials/floodfill-bootstrap/}

Dies würde es auch erlauben ein Testnetzwerk zu erstellen, deren Knoten nicht von allen anderen Knoten Bescheid wissen.


Um bessere empirische Aussagen machen zu können, sollte das Testnetzwerk so gut wie möglich das öffentliche I2P-Netzwerk abbilden \seereq{EVAL}.


\section{Messen der Ressourcenauslastung der Knoten}

Die Messung der Ressourcenauslastung der Knoten könnte sinnvoll sein aus mehreren Gründen.
Gerät ein Knoten oder die Hostsysteme für das Testnetzwerk in Ressourcenengpässe beeinflusst dies wahrscheinlich auch die Resultate.

\begin{itemize}
    \item Somit kann sichergestellt werden, dass das Testnetzwerk nicht so gross skaliert wurde, bzw. wie weit es skaliert werden kann \seereq{TSCL}.
    \item Um sicherzustellen das ein Flaschenhals im Netzwerk nicht durch zu wenig Ressourcen verursacht sind \seereq{TPER}.
    \item Äussere einflüsse wie Ressource-Exhaustion vermeiden, dies verbessert auch die Reproduzierbarkeit \seereq{TREP}.
    \item Messung der Test-Performance
\end{itemize}

Gemessen werden sollte:

\begin{itemize}
    \item Netzwerkbandbreite (kbits/sec)
    \item CPU gebrauch (in Prozent)
    \item RAM Verwendung (in Prozent)
    \item Disk I/O (read/write, kbits/sec)
    \item Freier Speicherplatz
\end{itemize}
