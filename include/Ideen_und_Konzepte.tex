\chapter{Ideen und Konzepte}
\section{Grundidee}
% TODO Beschreibung wie das Problem im Ansatz gelöst werden soll

% Hier geht es um die Fragestellung, wie Sie die formulierten Ziele der Arbeit erreichen wollen.
% Sie halten z.B. erste, grobe Ideen, skizzenhafte Lösungsansätze fest. Gibt es mehrere Wege, Ansätze
% um dieses Ziel zu erreichen, begründen Sie hier, warum Sie einen bestimmten Weg einschlagen.
% Beispiel für ein Softwareprojekt: Erste Gedanken über eine grobe Systemarchitektur. Ist z.B. eine
% Microservice-Architektur angebracht? Welche Alternativen bestehen, wo gibt es Problempunkte? Die
% Umsetzung, die Beurteilung der Machbarkeit und die detaillierte Beschreibung der umgesetzten
% Architektur sind dann Teil der Realisierung.

% Abgrenzung zu Kapitel 5 (Realisierung):
% - Besteht ein wesentliches Projektziel darin, für Ihre Kunden z.B. ein Security-Konzept, ein
% Kommunikations-Konzeptes, ein IT-Fachkonzept oder ein anderes Fach-Konzept zu erstellen, dann
% wird die Entwicklung dieser (fachlichen) Konzepte unter «Realisierung» beschrieben (sie sind ja der
% eigentliche Kern Ihrer Arbeit).
% - Besteht z.B. ein wesentliches Ziel der Arbeit darin, eine passende Software-Architektur zu
% evaluieren, dann gehören die entsprechenden Beschreibungen ins Kapitel 5.

\section{Reproduzierbatkeit}

\begin{itemize}
    \item Isoliertes Netzwerk
    \item Configuration as Code
    \item Nix
    \item Container
\end{itemize}

\section{Deployment-Tool}

\begin{itemize}
    \item Mehrere Maschinen auf einmal
    \item Reproduzierbatkeit
    \item verschiedene Konfigurationsänderungen
    \item Verschiedene Deployment möglichkeiten (HW/VM/Container)
\end{itemize}

\section{Latenzmessung von Nachrichten}

\begin{itemize}
    \item (Empfangszeitpunkt - Absendezeitpunkt)
    \item Clock Synch?
\end{itemize}

\section{Limitieren der Bandbreite}

\begin{itemize}
    \item Die Bandbreite eines i2pd Knoten kann limitiert werden in der Konfiguration mit der Einstellung bandwidth.  
    \item \cite{noauthor_i2p_nodate-3}
    \item Tc - Traffic control tool
    \item Nix Tooling
\end{itemize}

\section{Isolierung des Testnetzwerks}

\begin{itemize}
    \item Firewall
    \item Container
    \item VM Network?
    \item Segmentierte Netzwerke?
\end{itemize}

\section{Messen der Ressourcenauslastung der Knoten}

\begin{itemize}
    \item Uplink/Downlink Bandwith
    \item Bandwidth of I2P traffic
    \item CPU usage
    \item RAM Usage
\end{itemize}
