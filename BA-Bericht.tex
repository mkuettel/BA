\documentclass[
    a4paper,
    dvipsnames, % more color names with xcolor
]{scrreprt}

%%% PACKAGES %%%

% PDF/A Compliance
\usepackage[a-2b,latxmp]{pdfx}
\catcode30=12

% add unicode support and use german as language
\usepackage[utf8]{inputenc}
\usepackage[ngerman]{babel}

% Use Helvetica as font
\usepackage[scaled]{helvet}
\renewcommand\familydefault{\sfdefault}
\usepackage[T1]{fontenc}

% Better tables
\usepackage{tabularx}
\usepackage{xcolor} % more colors
\usepackage{colortbl} % color cells
\usepackage{booktabs} % \midrule & \toprule & \bottomrule
\usepackage{multirow} % multi-row cells in tables
% reference table rows
% https://tex.stackexchange.com/questions/356815/reference-table-row-in-latex
\newcounter{rowcntr}[table]
\renewcommand{\therowcntr}{\arabic{rowcntr}}
% A new columntype to apply automatic stepping
\newcolumntype{N}{>{\refstepcounter{rowcntr}\therowcntr}l}
% required for \AtBeginEnvironment
\usepackage{etoolbox}
% Reset the rowcntr counter at each new tabular
\AtBeginEnvironment{tabular}{\setcounter{rowcntr}{0}}


% Use have more float options
\usepackage{float}
% float figures left/right
\usepackage{wrapfig}

% Better enumerisation env
\usepackage{enumitem}

% Use graphics
\usepackage{graphicx}

% Have subfigures and captions
\usepackage{subcaption}

% Be able to include PDFs in the file
\usepackage{pdfpages}

% Have custom abstract heading
\usepackage{abstract}

% Need a list of equation
\usepackage{tocloft}
\usepackage{ragged2e}

% Better equation environment
\usepackage{amsmath}

% Symbols for most SI units
\usepackage{siunitx}

\usepackage{csquotes}
% be able to include source code listings
\usepackage{listings}
\usepackage{listingsutf8}
\usepackage{inconsolata}
\definecolor{dkgreen}{rgb}{0,0.5,0}
\definecolor{dkblue}{rgb}{0,0,.6}
\definecolor{dkyellow}{cmyk}{0.2,0,0.8,0.3}
\lstset{
    basicstyle         = \footnotesize{\ttfamily},         % Code font, Examples: \footnotesize, \ttfamily
    breaklines         = true,                           % automatic line breaking only at whitespace
    captionpos         = b,                              % sets the caption-position to bottom
    backgroundcolor    = \color{white},                  % Choose background color
    frame              = tb,                         % A frame around the code
    tabsize            = 4,                              % Default tab size
    captionpos         = b,                              % Caption-position = bottom
    % postbreak          = \mbox{$\hookrightarrow$\space},
    showspaces         = false,                          % Dont make spaces visible
    showtabs           = false,                          % Dont make tabs visible
    inputencoding      = utf8,
    extendedchars      = true,                           % i have some special chars...
    mathescape         = false,                          % i do not like fancy formulas
    escapeinside       = {},                             % i do not like to escape something
    escapechar         = {},                             % so i have no char defined
    texcl              = false,                          % and i do not wanna make fancy comments too..
    numbers            = left,
    language           = php,
    numberstyle        = \ttfamily,
    basicstyle         = \small\ttfamily,
    keywordstyle       = \color{dkblue},
    stringstyle        = \color{red},
    identifierstyle    = \color{black},
    commentstyle       = \color{gray},
    % emph               = [1]{php,if,and,or,else,for,if,foreach,as,function,echo,break,continue,switch,case},
    % emphstyle          = [1]\color{dkyellow},
    % emph               = [2]{public,class,abstract,private,protected,interface,use,trait,namespace,static,this,$this,extends,implements},
    % emphstyle          = [2]\color{dkblue},
    % emph               = [3]{int,string,array,mixed,float,double},
    % emphstyle          = [4]\color{dkgreen},
}

% YAML style in listings
\newcommand\YAMLcolonstyle{\color{red}\mdseries}
\newcommand\YAMLkeystyle{\color{black}\bfseries}
\newcommand\YAMLvaluestyle{\color{blue}\mdseries}

% Hypothesen
\usepackage{ntheorem}
\theoremseparator{:}
\newtheorem{hyp}{Hypothese}

% clickable links to websites and chapters
\usepackage{hyperref}
\hypersetup{
  colorlinks   = true, %colours links instead of ugly boxes
  urlcolor     = RoyalPurple, %colour for external hyperlinks
  linkcolor    = black, %colour of internal links
  citecolor    = black, %colour of citations
  breaklinks   = true
}

% shortcut to create refrences
\newcommand*{\fullref}[1]{\hyperref[{#1}]{\ref*{#1} ``\nameref*{#1}''}}
\newcommand*{\reqref}[1]{\hyperref[{req:#1}]{\ref{req:#1}-#1}}
\newcommand*{\seereq}[1]{(siehe Anforderung \reqref{#1})}
\newcommand*{\rid}[1]{\label{req:#1}-#1}

% Change page rotation
\usepackage{pdflscape}

% Symbols like checkmark
\usepackage{amssymb}
\usepackage{pifont}

\usepackage[absolute]{textpos}

% Glossary, hyperref, babel, polyglossia, inputenc, fontenc must be loaded before this package if they are used
\usepackage{glossaries}
% Redefine the quote charachter as we are using ngerman
\GlsSetQuote{+}
% Define the usage of an acronym, Abbreviation (Abbr.), next usage: The Abbr. of ...
\setacronymstyle{long-short}

% Display CSV files as table
\usepackage{csvsimple}

% making diagrams
\usepackage{tikz}
\usetikzlibrary{arrows,automata}
\usetikzlibrary{positioning}
\tikzset{>=latex}

% Bibliography & citing
\usepackage[
	backend=biber,
	style=apa,
	bibstyle=apa,
	citestyle=apa,
	sortlocale=de_DE
	]{biblatex}
\addbibresource{Referenzen.bib}
\DeclareLanguageMapping{ngerman}{ngerman-apa}

% Clickable Links to Websites and chapters
% From the documentation: "Make sure it comeslastof your loaded packages, to give it a fighting chance of not beingover-written, since its job is to redefine many LATEX commands"
\usepackage{hyperref}

%%% COMMAND REBINDINGS %%%
\newcommand{\tabitem}{~~\llap{\textbullet}~~}
\newcommand{\xmark}{\ding{55}}
\newcommand{\notmark}{\textbf{\textasciitilde}}
% Pro/Con item https://tex.stackexchange.com/questions/145198/change-the-bullet-of-each-item#145203
\newcommand\pro{\item[$+$]}
\newcommand\con{\item[$-$]}

%%% TRAVIS DEPENDENCY
% Can delete if you don't use Travis for CI/CD
% Fallback DocumentVersion for the preamble if built locally
\providecommand{\docversion}{1.0}

% Define list of equations - Thanks to Charles Clayton: https://tex.stackexchange.com/a/354096
\newcommand{\listequationsname}{\huge{Formelverzeichnis}}
\newlistof{myequations}{equ}{\listequationsname}
\newcommand{\myequations}[1]{
    \addcontentsline{equ}{myequations}{\protect\numberline{\theequation}#1}
}
\setlength{\cftmyequationsnumwidth}{2.3em}
\setlength{\cftmyequationsindent}{1.5em}

% Usage {equation}{caption}{label}
% \indexequation{b = \frac{\pi}{\SI{180}{\degree}}\cdot\beta\cdot 6378.137}{Bogenlänge $b$ des Winkels $\beta$ mit Radius 6378.137m (Distanz zum Erdmittelpunkt am Äquator)}{Bogenlaenge}
\newcommand{\indexequation}[3]{
	\begin{align} \label{#3} \ensuremath{\boxed{#1}} \end{align}
	\myequations{#3} \centering \small \textit{#2} \normalsize \justify }

% Todolist - credit to https://tex.stackexchange.com/questions/247681/how-to-create-checkbox-todo-list
\newlist{todolist}{itemize}{1}
\setlist[todolist]{label=$\square$}

% Nested Enumeratelist credit to https://tex.stackexchange.com/a/54676
\newlist{legal}{enumerate}{10}
\setlist[legal]{label*=\arabic*.}

%%% PATH DEFINITIONS %%%
% Define the path were images are found
\graphicspath{{./img/}{./appendix/}}


%%% GLOSSARY ENTRIES %%%
\makeglossaries
% \newacronym{RFID}{RFID}{Radio-Frequency Identification}
% \newglossaryentry{HF}{name={HF},description={High Frequency, RFID Tags im Frequenzbereich von 3-30MHz}}
\newglossaryentry{p2p}                 { name={Peer-to-Peer},                  text={Peer-to-Peer},           description={{
    Peer-to-Peer ist
    %TODO: define Peer-to-Peer
}}}
\newglossaryentry{refactoring}         { name={Refactoring},                   text={refactoring},            description={{
    ``The process of changing a software system in such a way that it does
        not alter the external behavior of the code yet improves its internal
        structure'' ---\cite{fowler_refactoring_2018}
}}}
\newglossaryentry{ci}                  { name={Continuous Integration},        text={continuous integration}, description={{
    ``Continuous Integration is a software development practice where members
    of a team integrate their work frequently,                                 usually each person integrates
    at least daily - leading to multiple integrations per day. Each integration
    is verified by an automated build (including test) to detect integration
    errors as quickly as possible.'' -- \cite{fowler_continuos_2014}
}}}
\newglossaryentry{pr}                  { name={Pull Request},                  text={pull request},           description={{
    Eine Funktion von Code-Hosting Plattformen und ein Prozess worin ein Entwickler einen andern anfragen kann ob er Änderungen in sein Software-Projekt aufnehmen will.
}}}
\newglossaryentry{saas}                { name={SaaS},                          text={Software as a Service},  description={{
            \glstext{saas} ist meist zentral gehostete Software welche Abonnemente für Ihre Kunden anbietet.
    For Example: Web-Mail, Dropbox
}}}
\newglossaryentry{dht}                 { name={DHT},        text={Distributed Hashed Table},                    description={{
    Eine \glstext{dht} ist eine auf verschiedenen Knoten oder Rechnern verteilte Key-Value Datenbank.
}}}
\newglossaryentry{e2e}                 { name={E2E},        text={End-to-End Encryption},                     description={{
    Von Endgerät zu Endgerät verschlüsselt. Das heisst ein Man-in-the-Middle Angreifer könnte nur verschlüsselten Traffic mitlesen.
}}}
\newglossaryentry{tor}                 { name={TOR},        text={The Onion Router},                          description={{
    Von Endgerät zu Endgerät verschlüsselt. Das heisst ein Man-in-the-Middle Angreifer könnte nur verschlüsselten Traffic mitlesen.
}}}
\newglossaryentry{i2p}                 { name={I2P},        text={The Invisible Internet Protocol},                          description={{
            Dezentrales nachrichtenorientiertes Mischnetzwerk indem anonym verschlüsselte Nachrichten ausgetauscht werden können \parencite[p.~1]{zantout_i2p_2011}. Die Abkürzung I2P wird auch als the Invisible Internet Project ausgelegt, um das gesamte Projekt nicht nur das Protokoll zu beschreiben.
}}}
\newglossaryentry{nixos-container}    { name={}, text={}, description={{{
    Containertechnologie ähnlich wie docker.
    Wichtig dabei ist, dass es sich hierbei nicht um einen Docker-Container, sondern um NixOS-Container handelt technologie handelt, welche auf systemd-nspawn Containern basiert..
    Im Hintergrund brauchen jedoch beide Container-Arten die gleichen Linux-Kernel-Features (namespaces, cgroups, ... ).
}}}
% TODO: add vm / container here?


%%% DOCUMENT %%%

\begin{document}

% !TEX root = BA-Bericht.tex
\begin{titlepage}
	\begin{textblock*}{5cm}[0,0](15.1cm,0.7cm)
		\includegraphics[keepaspectratio,width=5cm]{img/HSLU_Logo}
	\end{textblock*}
	\begin{center}
		\vspace*{5cm}
        \Huge{\textbf{Untersuchung der Performanz des Invisible Internet Protocols (I2P)} \\
		\vspace{0.5em}
		\Large{Bachelorarbeit FS2021}\\
		\vspace{3em}
		\LARGE{Studentin / Student}\\
		\LARGE{Moritz Küttel}\\
		\vspace{1em}
		\Large{Betreuer:}\\
        \large{Dr. Dieter Arnold}
		\vfill
		\large{Hochschule Luzern - Departement Informatik}\\
		\large{\today}\\
		\large{Version \docversion}
	\end{center}
\end{titlepage}

\newpage

\pagenumbering{gobble}

\begin{textblock*}{5cm}[0,0](15cm,0.7cm)
	\includegraphics[keepaspectratio,width=2.7cm]{img/HSLU_Logo_Header}
\end{textblock*}

\vspace*{1.35cm}

\fontsize{12}{14}
\textbf{Bachelorarbeit an der Hochschule Luzern -- Informatik} \\ \vspace*{0.6cm}

\fontsize{10.5}{12}
\noindent
\textbf{Titel:} Untersuchung der Performance des Invisible Internet Protocols (I2P)\newline \newline
%TODO: HSLU Preamble titel
\textbf{Studentin/Student:} Moritz Küttel \newline \newline
\textbf{Studiengang:} BSc Informatik \newline \newline
\textbf{Jahr:} 2021\newline \newline
\textbf{Betreuungsperson:} Dr. Dieter Arnold \newline \newline
\textbf{Expertin/Experte:} Urs Rufer \newline \newline
\textbf{Auftraggeberin/Auftraggeber:} Carolyn Bächler, Konrad Bächler, DIVA.EXCHANGE\newline \newline \newline
\textbf{Codierung / Klassifizierung der Arbeit:}\\
$\boxtimes$ A: Einsicht   (Normalfall) \\
$\square$ B: R\"ucksprache      (Dauer:  \ \ \ \ \     Jahr / Jahre)\\
$\square$ C: Sperre     (Dauer:  \ \ \ \ \     Jahr / Jahre)\\

\paragraph{\textbf{Eidesstattliche Erkl\"arung}}\\
Ich erkl\"are hiermit, dass ich/wir die vorliegende Arbeit selbst\"andig und ohne unerlaubte fremde Hilfe angefertigt haben, alle verwendeten Quellen, Literatur und andere Hilfsmittel angegeben haben, w\"ortlich oder inhaltlich entnommene Stellen als solche kenntlich gemacht haben, das Vertraulichkeitsinteresse des Auftraggebers wahren und die Urheberrechtsbestimmungen der Fachhochschule Zentralschweiz (siehe Merkblatt <<Studentische Arbeiten>> auf MyCampus) respektieren werden. \newline \newline
Zürich, den 19. September 2021, Unterschrift       \underline{\hspace*{4cm}} \newline \newline


\newpage
\noindent
\textbf{Abgabe der Arbeit auf der Portfolio Datenbank:}\\
Best\"atigungsvisum Studentin/Student\\
Ich best\"atige, dass ich die Bachelorarbeit korrekt gem\"ass Merkblatt auf der Portfolio Datenbank abgelegt habe. Die Verantwortlichkeit sowie die Berechtigungen habe ich abgegeben, so dass ich keine \"Anderungen mehr vornehmen kann oder weitere Dateien hochladen kann. \newline \newline
Zürich den 19. September 2021, Unterschrift       \underline{\hspace*{4cm}} \newline \newline
\noindent
{\textbf{Ausschliesslich bei Abgabe in gedruckter Form: \\
Eingangsvisum durch das Sekretariat auszuf\"ullen}} \newline \newline
Rotkreuz, den   \underline{\hspace*{4cm}} \hspace*{1cm} Visum:  \underline{\hspace*{4cm}} 
\newline \newline \newline
\textbf{Verdankung}\\
An dieser Stelle möchte ich mich bei allen bedanken, die mich bei dieser Bachelorarbeit unterstützt haben.

Als allererstes möchte ich meinem Betreuer Dieter Arnold danken, der meine Bachelorarbeit betreut und begutachtet hat und stets für meine Fragen und Anliegen da war. Insbesondere möchte ich mich für seine aussergewöhnliche Unterstützung während dieser schwierigen Zeit bedanken.

Ich bedanke mich bei Konrad und Carolyn Bächler vom Verein DIVA.EXCHANGE, für diese tolle Projektidee, und für die hervorragende Arbeitsumgebung in ihrem Büro in Baar. Bedanken möchte ich mich auch für die vielen Diskussionen und Gespräche sowie Ideen und Anregungen, welche diese Arbeit in eine bessere Richtig gelenkt haben.

Ausserdem möchte ich meiner Mutter Bernadette Lüönd, sowie meinem guten Freund Moritz Köhler danken für das Korrekturlesen meiner Arbeit danken und dafür dass sie immer ein offenes Ohr für mich hatten.
Ebenfalls möchte ich mich bei meiner guten Freundin Linda Hauser bedanken für ihren Rückhalt, sowie für die Hilfe beim erstellen des Pitching-Videos.
\newline \newline \newline
\newline \newline \newline
\newline \newline \newline
\newline \newline \newline
{\textbf{Hinweis}}: Die Bachelorarbeit wurde von keinem Dozierenden nachbearbeitet. Ver\"offentlichungen (auch auszugsweise) sind ohne das Einverst\"andnis der Studiengangleitung der Hochschule Luzern -- Informatik nicht erlaubt.
\newline
Copyright \textcopyright\ 2021 Hochschule Luzern -- Informatik \newline \newline
Alle Rechte vorbehalten. Kein Teil dieser Arbeit darf ohne die schriftliche Genehmigung der Studiengangleitung der Hochschule Luzern -- Informatik in irgendeiner Form reproduziert oder in eine von Maschinen verwendete Sprache \"ubertragen werden.

\newpage
\pagenumbering{arabic}



\KOMAoptions{headings=normal}
\KOMAoptions{DIV=12}
\pagenumbering{Roman}


\begin{abstract}


Der Verein DIVA.EXCHANGE entwickelt einen Software-Prototypen für eine Handelsplattform, um digitale Werte auszutauschen.
Die Handelsplattform soll \glstext{fullydistributed} sein und es sollen
digitale Werte ausgetauscht werden können, ohne dass sich die Benutzer gegenseitig kennen oder vertrauen zu müssen.
Um dies umzusetzen, wurde auf das Anonymisierungsnetzwerk ``The Invisible Internet Protocol'' (I2P) als Grundstein auf der Netzwerkebene gesetzt.
Jedoch hat die Anonymität, Privatsphäre und Sicherheit, die durch I2P geboten wird, ihren Preis: Performanz.
In dieser Arbeit soll empirisch untersucht werden, unter welchen Umständen sich die Performanz des I2P-Netzwerks verbessert.
Insbesondere wurden die Latenzzeiten von TCP-Nachrichten untersucht.

Damit die Performance-Messungen empirisch durchgeführt werden konnten, wurde ein Teststand entwickelt an dem Experimente durchgeführt wurden.
In einem privaten I2P-Netzwerk konnten Latenzmessungen ohne äussere Einflüsse durchgeführt werden.
Damit konnte ein Testnetzwerk bestehend aus bis zu 256 I2P-Knoten erstellt werden.
Somit konnten Latenzmessungen durchgeführt werden, welche aufgezeigt haben,
das sich entgegen unserer Erwartungen die Latenz bei einem grösseren Netzwerk nicht unbedingt verbessert.
Die genauen Gründe sind jedoch unklar.

Die gewonnenen Resultate dienen jedoch als Grundlage und Referenz für zukünftige Messungen und Experimente.
Anhand des entwickelten Teststands können nun weitere Messungen getätigt und miteinander verglichen werden.
Grundsätzlich können so auch verschiedene I2P-Konfigurationen miteinander verglichen werden.
Es sind aber in Zukunft auch Tests mit anderen Protokollen oder Nachrichtentypen möglich.

\end{abstract}

\tableofcontents

\newpage

\clearpage
\pagenumbering{arabic}

\chapter{Einleitung}
\label{ch:Einleitung}
% TODO Allgemeine Einleitung in die Arbeit

\section{Aufgabenstellung und Zielsetzung}

Siehe auch die komplette Aufgabenstellung im Anhang \fullref{ch:aufgabenstellung}

\section{Problem}

Man hat festgestellt das Aplikationen auf dem I2P-Netzwerk für Endbenuzter langsam erscheinen.
Damit Applikationen auf dem I2P-Netzkwerk gut funktioniern, soll ermittelt werden wie die Performance
verbesswert werden soll.

\section{Fragestellung}

Verringert sich die Latenz im I2P Netzwerk je mehr Knoten es gibt?
Verringert sich die Latenz im I2P Netzwerk je mehr Bandbreite es gibt?

\section{Vision}

Wird in dieser festgestellt, das mehr Knoten oder verschiedene Arten von Knoten, könnte man mit diese Erkenntnissen in i2p-Netzkwerk aufnehmen.
Diese Erkenntnis könnte auch unter umständen mehr Personen dazu bewegen selber i2p-Knoten zu betreiben.

% Welche Ziele, Fragestellungen werden mit dem Projekt verfolgt? Die Bedeutung, Auswirkung und
% Relevanz dieses Projektes für die unterschiedlichen Beteiligten soll aufgeführt werden.
% Typischerweise wird hier ein Verweis auf die Aufgabenstellung im Anhang gemacht.


% !TEX root = BA-Bericht.tex
\chapter{Stand der Technik}\label{ch:StandDerTechnik}
% Auch "Stand der Forschung" oder "Stand der Praxis"

% Bezogen auf die eigenen Zielsetzungen und Fragestellungen soll aufgezeigt werden, wie andere
% dieses oder ähnliche Probleme gelöst haben. Worauf können Sie aufbauen, was müssen Sie neu
% angehen? Wodurch unterscheidet sich Ihre Lösung von anderen Lösungen? Für wissenschaftlich
% orientierte Arbeiten sei hier explizit auf (Balzert, S. 66 ff) verwiesen.


% TODO Historie des betroffenen Feld


\section{Technologische Grundlagen}
% TODO Wichtige technologischen Grundlagen / Wissenswertes

\subsection{Anonymisierungs-Netzwerke}

Es gibt eine vielzahl an verschiedenen Anonymisierungsnetzwerken, wobei \glstext{tor} (\glsname{tor}) wahrscheinlich das bekannteste davon ist.

In dieser Arbeit wird aber lediglich das Anonymisierungsnetzwerk \glsname{i2p} behandelt und nicht weiter auf andere Anonymisierungsnetzwerke eingegangen, da dieses das Netzwerklayer von diva.exchange darstellt.

\subsection{Anonymität, Privatsphäre und Datenschutz}

 Anonymität und Privatsphäre ist nicht das gleiche.
 Jedoch kann durch Anonymität die Privatsphäre geschützt werden.


 Auch ist Anonymität wichtig für Whistle-Blower und für Leute die unter totalitären Regimen.

\subsection{Das Anonymitätstrilemma}\label{sec:anonymitytrilemma}

Beim Design eines anonymen Kommunikationsprotokoll gibt es grundsätzliche Einschränkungen. Es gilt die folgenden drei Aspekte abzuwägen :

\begin{itemize}
    \item Der Anonymitätsgrad von Sender und Empfänger einer Nachricht
    \item Die benötigte Netzwerkbandbreite
    \item Die Latenzzeiten der Nachrichten
\end{itemize}


Dabei ist es nicht möglich ein Protokoll zu designen, welches einen hohen Anonymitätsgrad bietet, wenig Netzwerkbandbreite benötigt und tiefe Latenzzeiten vorweist.
Die Anonymität geht also immer auf Kosten von höherer Latenzzeiten und/oder mehr benötigter Netzwerkbandbreite (siehe auch Abbildung~\fullref{fig:anonimitytrilemma}).

\begin{figure}[H]
    \centering
    \begin{tikzpicture}
      \draw (0,0) node[anchor=north]{hoher Anonymitätsgrad}
        -- (4,0) node[anchor=north]{tiefe Latenzzeiten}
        -- (2,4) node[anchor=south]{wenig Netzwerkbandbreite}
        -- cycle;
    \end{tikzpicture}
    \caption{Das Anonymitätstrilemma}\label{fig:anonimitytrilemma}
\end{figure}

Ein Anonymisierungsnetzwerk kann beispielsweise auf Kosten der Netzwerkbandbreite implementiert werden, indem Nachrichten einfach an alle im Netzwerk versendet werden. Somit könnte mindestens der Empfänger verschleiert werden.
Umgekehrt könnte ein Anonymisierungsnetzwerk auf Kosten der Latenz implementiert werden. Wenn Nachrichten über mehrere Netzwerkknoten geleitet werden, könnte man Empfänger und Sender verschleiern, aber hätte höhere Latenzzeiten da dieselbe Nachricht mehrmals nacheinander von verschiedenen Netzwerkknoten bearbeitet werden muss.
Letzteres gibt es Netzwerke bei welchen Anonymität keine Rolle spielt, wie beispielsweise beim üblichen UDP Protokoll. Es ist dann möglich hohe Bandbreite sowie niedrige Latenzzeiten zu erreichen.

Es gilt abzuwägen, welche der Aspekte für das zu designende Protokoll wichtig sind und wo ein Schwerpunkt gelegt wird und welche Aspekte somit auch vernachlässigt werden, um einen guten Kompromiss zu finden. \parencite{das_anonymity_2018}

\subsection{\glstext{i2p} (\glsname{i2p})}

Kurz gesagt ist \glstext{i2p} (\glsname{i2p}) ein dezentrales Mischnetzwerk mit niedriger Latenz indem verschlüsselte Nachrichten ausgetauscht werden können.
\parencite{zantout_i2p_2011}
Es handelt sich hier um ein eigenständiges Overlay-Netzwerk, welches auf den darunterliegenden Protokollen UDP und TCP aufbaut.  \parencite{de_boer_invisible_2019,astolfi_i2p_2015}.

Mit einem Mischnetzwerk ist gemeint, dass Nachrichten an durch Knoten des Netzwerks weitergeleitet werden, um den Sender und den Empfänger einer Nachricht zu verschleiern. Dieser Mischvorgang wird bei I2P Garlic-Routing genannt.
\parencite[S.~1]{zantout_i2p_2011}.

Auch gilt es zu unterstreichen, dass \textit{Nachrichten} ausgetauscht werden.
I2P ist \textit{Nachrichtenbasiert} und verhält sich in diesem Sinne ähnlich wie UDP.
Es gibt dementsprechend keine Garantien, dass wenn eine Nachricht versendet wird, dass die Daten in der richtigen Reihenfolge, komplett oder überhaupt ankommen.
Es gibt jedoch Erweiterungen auf dem Netzwerk, welche diese Garantien (ähnlich wie sie diese bei TCP gibt) wieder herstellen können.
Zudem bietet I2P eine Ende-zu-Ende-Verschlüsselung für die Nachrichten. Die Nachrichten sind also vom Sender bis zum Empfänger verschlüsselt.

Während das TOR-Netzwerk ursprünglich entwickelt wurde auf das öffentliche Internet via Out-Proxies zuzugreifen,
liegt bei I2P der Fokus darin anonym auf Dienste, die innerhalb des Netzwerks angeboten werden, zuzugreifen.
Jedoch besteht bei TOR ebenfalls die Möglichkeit Netzwerkinterne Dienste anzubieten mittels TOR Hidden Services, auch aber gibt es bei I2P die Möglichkeit, mit Hilfe von sogenannten Outproxies, auf das normale Internet zuzugreifen.

Dienste die innerhalb des Netzwerks angeboten werden, sind zum Beispiel Bit-Torrent für File-Sharing, Webseiten (sogenannte ``eepsites''), oder auch Chat-Dienste wie IRC (Internet Relay Chat).
\parencite[p.~3-4]{de_boer_invisible_2019}

\cite{astolfi_i2p_2015}

\cite{timpanaro_birds_2012}

\cite{timpanaro_evaluation_2015}

A scaleable framework for anonymous communication
\cite{noauthor_i2p_nodate-8}


\cite{hoang_measuring_2019}
\cite{hoang_empirical_2018}

\cite{de_boer_invisible_2019}

\cite{zantout_i2p_2011}

Latency vs Tor Usability Bandwith and Latency Comparision
\cite{ehlert_i2p_2021}

\subsection{Wie I2P funktioniert}

\begin{itemize}
    \item Jeder Knoten leitet auch traffic für das Netzwerk weiter
    \item Jeder Knoten kann die länge 
\end{itemize}

\section{Technische Konzepte}
\label{sec:technischeKonzepte}
% TODO Konzepte in diesem Feld welche für den Leser relevant sind

\subsection{Tunnels}

Jeder I2P-Router kann eine beliebige Anzahl Inbound- und Outbound Tunnels erstellen.
Wobei die Inbound Tunnels zum Empfang und die Outbound-Tunnels zum Versand dienen.

%TODO: inbound outbound nicht verwechseln

Jedes Tunnel hat eine Länge.
Diese gibt an durch wie viele Knoten eine Nachricht, die durch das Tunnel verschickt wird, geroutet wird.
Für jeden Hop im Tunnel wird die Nachricht eine zusätzliche Verschlüsselungsschicht eingehüllt. (vgl. Onion Routing)
Somit wissen die Router jeweils nur Bescheid von welchem anderen Router sie eine Nachricht erhalten haben und an welchen Router die Nachricht weitergeleitet werden soll.
Ein Roter im Tunnel weiss also schlussendlich nie von wem an wen eine Nachricht verschickt wurde.

Die Länge eines Tunnels kann auch \lstinline|0| betragen.
In diesem Fall wird eine Nachricht die durch das Tunnel verschickt wird, durch keine weitere Knoten geroutet.

Hierbei kann jeder Knoten selber festlegen wie lange diese Tunnels sind.

Je Länger ein Tunnel ist, desto mehr Anonymität bietet dieser.
Jedoch geht dies auf Kosten der Latenz aufgrund der zusätzlichen Hops.

Tunnels Configuration
\cite{noauthor_i2p_nodate-3}

\begin{itemize}
    \item Jeder Knoten kann die länge seiner Tunnel selber bestimmen. (Standardwert 3)
    \item Mehrere Verschlüsslungslayer je nach Tunnellänge (vgl. Onion-Routing)
\end{itemize}

\subsection{Garlic-Routing}

Im Gegensatz zum \glsname{tor}-Netzwerk setzt

\begin{itemize}
    \item
\end{itemize}

\subsection{NetDB}

\subsection{Reseeding}

Seeding etc.

Reseed Access
\cite{noauthor_i2p_nodate-7}


\section{Technische Konzepte}
\label{sec:technischeKonzepte}
% TODO Konzepte in diesem Feld welche für den Leser relevant sind

\subsection{Das Anonymitätstrilemma}\label{sec:anonymitytrilemma}

Beim Design eines anonymen Kommunikationsprotokoll gibt es gründsätzliche Einschränkungen. Es gilt die folgenden drei Aspekte abzuwägen:

\begin{itemize}
    \item Hoher Anonymiätsgrad von Sender und Empfänger einer Nachricht
    \item Tiefer Netzwerkbandbreiten Overhead
    \item Tiefer Latenz Overhead
\end{itemize}

Dabei ist es nicht möglich ein Protokoll zu designen welches alle diese drei Aspekte komplett erfüllt.
Es muss entschieden werden, welche zwei Aspekte wichtiger sind (siehe auch die Abbildung~\fullref{fig:anonimitytrilemma}). \parencite{das_anonymity_2018}

\begin{figure*}[h]
    \begin{tikzpicture}
      \draw (0,0) node[anchor=north]{hohe Anonymität}
        -- (4,0) node[anchor=north]{tiefe Latenz}
        -- (2,4) node[anchor=south]{wenig Bandbreite}
        -- cycle;
    \end{tikzpicture}
    \caption{Das Anonymitätstrilemma}\label{fig:anonimitytrilemma}
\end{figure*}

Zum Beispiel wenn ein hoher Anonymitätsgrad erwünscht ist, kann dies nur erreicht werden mit entweder mehr Netzwerkbandbreiten-Overhead oder mehr Latenz-Overhead.


\section{Stand im Bezug auf eigenes Projekt}
% TODO Welche Forschung wurde in jüngster Zeit gemacht welche relevant für das eigene Projekt sind

Usability Inspection of Anonmity Networks
(\cite{abou-tair_usability_2009})

Usability Tetsts

(\cite{schomburg_anonymity_2009})

Evaluation of Anonymity Networks
(\cite{timpanaro_evaluation_2015})

\subsection{Performance}


Performance improvement using SSL IN I2P
\cite{vashi_performance_2015}

Performance I2P Webseite
\cite{noauthor_performance_nodate}

Performance History I2P Webseite
\cite{noauthor_performance_nodate-1}

Auf der Webseite von I2P auf der Seite ''Future Performance Improvements'' sind zudem zukünftige Performance Verbesserungsmöglichkeiten aufgelistet.
\cite{noauthor_future_nodate}
On I2P's website the page 

Improving I2P (2012)
\cite{timpanaro_improving_2012}


\subsection{I2P-Testnetzwerke}

\subsection{Metriken zur Messung}

Towards Measuring on the I2P Netzwork
\cite{wang_towards_2013}

\begin{itemize}
    \item Latenzmessung (abschicken/empfangen (Laport-Zeitstempel im schlimmsten fall))
    \item Messung der Bandbreite (ab welchem Layer)
    \item Ressourcenauslastung
\end{itemize}

\cite{timpanaro_monitoring_nodate}

\subsection{Testen im öffentlichen I2P Netzwerk}

\begin{itemize}
    \item family tag \cite{noauthor_family_nodate}
    \item verfälscht resultate
    \item deshalb isolieren
    \item öffentliche Metriken  I2P Metrics: \cite{noauthor_i2p_nodate-4}
    \item Netzwerk ist klein (deshalb braucht es so ein beweis wie hier)
\end{itemize}


% !TEX root = BA-Bericht.tex
\chapter{Ideen und Konzepte}
\label{ch:ideen_und_konzepte}

Dieses Kapitel zeigt Ideen auf, wie die Fragestellung angegangen wurde.
Es wird ein Konzept für einen Teststand vorgestellt, der den Anforderungen (siehe Abschnitt~\fullref{sub:Anforderungen}) gerecht wird.



% TODO Beschreibung wie das Problem im Ansatz gelöst werden soll

% Hier geht es um die Fragestellung, wie Sie die formulierten Ziele der Arbeit erreichen wollen.
% Sie halten z.B. erste, grobe Ideen, skizzenhafte Lösungsansätze fest. Gibt es mehrere Wege, Ansätze
% um dieses Ziel zu erreichen, begründen Sie hier, warum Sie einen bestimmten Weg einschlagen.
% Beispiel für ein Softwareprojekt: Erste Gedanken über eine grobe Systemarchitektur. Ist z.B. eine
% Microservice-Architektur angebracht? Welche Alternativen bestehen, wo gibt es Problempunkte? Die
% Umsetzung, die Beurteilung der Machbarkeit und die detaillierte Beschreibung der umgesetzten
% Architektur sind dann Teil der Realisierung.

% Abgrenzung zu Kapitel 5 (Realisierung):
% - Besteht ein wesentliches Projektziel darin, für Ihre Kunden z.B. ein Security-Konzept, ein
% Kommunikations-Konzeptes, ein IT-Fachkonzept oder ein anderes Fach-Konzept zu erstellen, dann
% wird die Entwicklung dieser (fachlichen) Konzepte unter «Realisierung» beschrieben (sie sind ja der
% eigentliche Kern Ihrer Arbeit).
% - Besteht z.B. ein wesentliches Ziel der Arbeit darin, eine passende Software-Architektur zu
% evaluieren, dann gehören die entsprechenden Beschreibungen ins Kapitel 5.

Grundsätzlich soll ein Teststand aufgebaut werden, an dem empirische Performance Messungen an einem I2P-Testnetzwerk durchgeführt werden können \seereq{TINF}.
Dabei gibt es verschiedene Herausforderungen und Aspekte die es zu beachten gilt.
In den folgenden Abschnitten werden nun Ideen vorgestellt und genauer erläutert, wie diese Herausforderungen angegangen werden.
In den folgenden Abschnitten werden verschiedene Lösungsansätze für verschiedene Teilprobleme und Aspekte vorgestellt.

%TODO: Hyptohesen kann man testen, bestätigen, neue hyptothesen & und oder neue Tests
%TODO: das empirische
%TODO: hat nichts mit I2P

%TODO: 2. isolieres netzwerk
%TODO: 3. technologie ideen / Realisierungsvariianten

\section{Reproduzierbarkeit}

Um Messungen auf einem solchen Testnetzwerk wiederholbar und so gut wie möglich reproduzierbar zu machen \seereq{TREP}, sind folgende zwei Ideen wichtig.
Als erstes soll das Testnetzwerk isoliert werden, um äussere Einflüsse durch das Netzwerk (auch das öffentliche I2P-Netzwerk) zu vermeiden (siehe auch Abschnitt~\ref{sec:isolierung}).
Zweiteres soll die Testinfrastruktur als Programmcode beschrieben werden (auch bekannt als \glstext{iac} (\glsname{iac})).
Somit kann dasselbe Testnetzwerk später wieder erneut wieder aufgebaut werden, anhand der versionierten Infrastrukturdefinition.
Mehr dazu im Abschnitt~\ref{sec:infrastructure}.

\section{Infrastruktur}\label{sec:infrastructure}

Wichtig für das Deployment der Infrastruktur für das Testnetzwerk ist, dass man schnell neue Messungen starten und neue Testnetzwerke erstellen kann \seereq{TPER}.
Dies auch mit der Möglichkeit verschiedene Konfigurationseinstellungen tätigen zu können \seereq{TCNF}.
Die im Vorherigen Abschnitt bereits erwähnt, soll das Deployment auch reproduzierbar sein \seereq{TREP}.
Es soll auch möglich sein verschieden grosse Testnetzwerke zu erstellen \seereq{TSCL}.

\subsection{Container oder Virtuelle Maschinen}

Um Ressourcen zu schonen und aufgrund von kleinerem Overhead sowie Startup-Time von Containern im Gegensatz zu virtuellen Maschinen sind diese wohl zu bevorzugen.
Zudem lassen sich Container schneller neu zu erstellen und zu deployen.
Container erlauben es auch schneller Tests durchzuführen \seereq{TPER}. 
aber auch Tests mit mehr Knoten zu machen, da weniger Ressourcen für einen einzelnen Knoten benötigt wird, da der Betriebssystemkernel geteilt wird.
Dies ist der Fall weil bei Containerlösungen (oder auch OS-Virtualisierung) im Gegensatz zu Virtuellen Maschinen der Betriebssystemkernel zwischen den Instanzen geteilt wird.

\subsection{Deployment}

%TODO: Dies ist eine realisierungsfrage

Es gibt verschiedene Tools für das Deployment von gesamten Netzwerken, virtuellen Maschinen oder Container erlauben. 
Diese sind z.B. Docker-Compose, Kubernetes, Teraform, oder auch NixOps.
Wobei bereits Erfahrungen gemacht wurden mit NixOps und Docker-Compose.
Aufgrund dessen wird eine Lösung diesen Tools angestrebt.
Diese Tools erlauben ein reproduzierbares Deployment.
Dies wird vor allem von NixOps als Feature angepriesen.

\section{Konfigurierbarkeit}

Nach Absprache mit dem Auftraggeber soll mindestens folgendes soll im Teststand konfigurierbar sein (\reqref{TCNF}):

\begin{itemize}
    \item Anzahl-Knoten im Testnetzwerk
    \item Maximale Bandbreite der einzelnen Knoten
    \item Anzahl Hops (Also die Länge der Tunnel)
    \item Ob im privaten oder öffentlichen Netzwerk getestet werden soll. (optional)
\end{itemize}

Nix als deklarative Konfigurationssprache erlaubt es beliebige Aspekte des Setups konfigurierbar zu machen.
Die meisten oben erwähnten Konfigurationsmöglichkeiten werden bereits durch verschiedene NixOS-Module (\lstinline|i2pd|, \lstinline|firewall|, ... ) angeboten.
Als anderer Ansatz kann eine JSON-Datei eingesetzt werden, da diese von vielen Programmiersprachen einfach eingelesen werden können.

\section{Latenzmessung}

Um die Latenz oder Verzögerung einer Nachricht im Testnetzwerks zu messen \seereq{TLAT}, könnte man jeweils den Empfangs- sowie Sendezeitpunkt einer Nachricht speichern und voneinander subtrahieren.

\indexequation{Latenz = Empfangszeitpunkt - Sendezeitpunkt}{Latenzberechnung}{Latenzberechnung}

Jedoch muss sichergestellt werden, dass die Uhrzeiten zwischen allen Knoten synchron ist.
Grundlage für eine saubere Latenzmessung ist zudem, dass man in keine Ressourcenengpässe gerät.
Dementsprechend ist das Überwachen der benötigen Ressourcen wichtig. Siehe dazu Abschnitt~\fullref{sec:ressourcenauslastung}.

\section{Limitieren der Bandbreite}

Die \lstinline|i2pd|-Software erlaubt es die Bandbreite eines Knotens \seereq{TLIM} mittels der \lstinline|bandwidth| Konfigurationseinstellung \seereq{TCNF} zu limitieren. \parencite{noauthor_i2p_nodate-3}
Zusätzlich gäbe es auch andere Wege die Bandbreite zu limitieren.
Eine Möglichkeit wäre dies auf Netzwerkebene mittels dem ``Traffic Control Tool'' (kurz \lstinline|tc| zu lösen.

\section{Isolierung des Testnetzwerks}
\label{sec:isolierung}

% äussere einflüsse / idealisierte umgebung
Um äussere Einflüsse zu vermeiden und eine idealisierte Testumgebung zu erschaffen muss das Testnetzwerk isoliert werden.
Das Diagramm~\fullref{fig:i2p-testnetwork} zeigt den Aufbau des Testnetzwerks auf.
Einerseits kann eine Isolation auf Ressourcenebene durch die verwendete virtuelle Maschine erreicht werden.
Andererseits stellt dies auch automatisch eine Netzwerkisolation zur Verfügung, je nach dem wie die Virtuelle Maschine mit dem Host verbunden wird.
Das Testnetzwerk soll standardmässig weder mit dem Internet verbunden, noch mit dem System des Testers, um äussere Einflüsse zu verringern und somit durch Isolation bessere Reproduzierbarkeit zu erreichen.

\begin{figure*}[ht]
  \includegraphics[width=1.0\textwidth]{i2p-testnetwork.png}
  \caption{I2P Testnetwork}\label{fig:i2p-testnetwork}
\end{figure*}

Bei der Firewall handelt es sich entweder um das Hostsystem (VM-Network) oder um das Testnetzwerk selber, welche die I2P-Knoten hostet.
Wichtig ist das der Tester, welcher das Netzwerk erstellt und die Tests durchführt sich auch ausserhalb des Testnetzwerkes befindet, um ungewollte Einflüsse zu vermeiden.
Die I2P-Knoten sind als Router dargestellt und befinden sich eigentlich alle in zwei Netzwerken gleichzeitig. Das normale IP Netzwerk und das I2P-Netzwerk.
Am Ende des Tests müssen die Testresultate an einem sicheren Ort ausserhalb des Testnetzwerks abgelegt werden, damit diese später analysiert werden können.

In der \lstinline|i2pd|-Konfiguration gibt es zusätzlich das \lstinline|familiy|-Tag, welches die Knoten als zusammenhängend markieren würde, würden diese aus Versehen mit dem echten I2P-Netzwerk reden. Dies ist auch interessant bezüglich Tests die am echten Netzwerk durchgeführt werden sollen.

%TODO: fullmesh?

\section{Bootstrapping}

Wie im Abschnitt~\fullref{sec:bootstrapping} erklärt wird im privaten Testnetzwerk ein Bootstapping-Vorgang benötigt.
Ganz zu Beginn, wenn ein Knoten dem I2P Netzwerk beitritt muss dieser wissen wo mindestens ein anderer Knoten ist.
Um dies zu bewerkstelligen wird ein separater Container dem Testnetzwerk hinzugefügt der einen Reseed-Server mit sich bringt.
Somit kann zu Beginn jeder Knoten eine Liste von anderen Knoten abfragen und so dem Netzwerk beitreten.
Anhand dieser Liste kann auch der Verbindungsgrad der Knoten gesteuert werden.
Man kann auch einen Floodfill-Router verwenden, um das Netzwerk zu bootstrappen \parencite{noauthor_bootstrapping_nodate}.
Jedoch scheint der Ansatz mit einem Reseed-Server schöner zu sein, da sich somit nicht extra Knoten im Netzwerk befindet, der die Messung beeinflussen könnte.

\section{Messen der Ressourcenauslastung}\label{sec:ressourcenauslastung}

%TODO: in kapitel latenz

Die Messung der Ressourcenauslastung der Knoten könnte aus mehreren Gründen sinnvoll sein.
Gerät ein Knoten oder die Hostsysteme für das Testnetzwerk in Ressourcenengpässe, beeinflusst dies auch die Resultate.
Dies ist insbesondere wichtig enorm wichtig bei Latenzmessungen, denn diese werden somit fehlerhaft und nicht reproduzierbar \seereq{TREP}.
Auch kann sichergestellt werden, dass das Testnetzwerk nicht so gross skaliert wurde, bzw. wie weit es skaliert werden kann \seereq{TSCL}.
Zudem kann die Test-Performance so gemessen werden \seereq{TPER}.
Folgendes könnte gemessen werden:

\begin{itemize}
    \item CPU Gebrauch (in Prozent)
    \item Netzwerkbandbreite (kbits/sec)
    \item RAM Verwendung (in Prozent)
    \item Disk I/O (read/write, kbits/sec)
    \item Freier Speicherplatz
\end{itemize}

Es wird erwartet, dass der Workload von I2P vor allem Netzwerklastig (weiterleiten über mehrere Knoten) und Prozessorlastig (Verschlüsselung) ist.
Da es sich hier um ein virtuelles Container-Netzwerk handelt, heisst dass die Last auf dem Netzwerk sich auch in der Prozessorlast zeigen wird.
Um die benötigte Anzahl Container zu starten und die I2P-Knoten zu betreiben wird auch viel Arbeitsspeicher benötigt.


\chapter{Methode}\label{ch:Methode}

% Hier halten Sie fest und begründen, welches Vorgehensmodell Sie für Ihr Projekt wählen. Sie
% verweisen allenfalls auf die daraus entstandenen, konkreten Terminpläne mit Meilensteinen, welche
% z.B. unter Realisierung (Kapitel 5) oder im Anhang versorgt sind.
% Bei Projekten mit einer verlangten wissenschaftlichen Tiefe werden hier die geplanten
% Forschungsmethoden wie quantitative/qualitative Interviews, Befragungen, Beobachtungen,
% Feldexperiment etc. beschrieben und begründet.
% Warum ist in Ihrer Situation ein Interview besser als eine Umfrage? Wer soll interview werden?
% 2(Sie können bei Bedarf in Absprache mit Ihrer Betreuungsperson dazu auch ein zusätzliches
% Methodencoaching beziehen).
% Bei Engineering-Projekten halten Sie weitere einzusetzende fachliche Methoden oder Techniken fest.
% Bei einem Softwareprojekt können dies z.B. der geplante Einsatz einer Anforderungsanalyse, der
% Einsatz von Review-Techniken (Architektur-Reviews) oder bekannter Programmiertechniken sein.
% Dazu gehört auch eine Teststrategie (wo setzen Sie im Projekt Schwerpunkte betr. Testen?). Die
% eigentliche Testdurchführung ist dann unter Realisierung, im Anhang oder einem selbstständigen
% Testdokument beschrieben.

% TODO Nach eigenen Bedürfnissen erweitern
% ------------------------------------- TEIL Wissenschaftliche Arbeiten ----------------------------------------------------
\subsection{Laborexperiment}
% quantitaiv, ehter verhaltenswissenschaftlich als konstruktiv

% Das Experiment untersucht Kausalzusammenhängein kontrollierter Umgebung, indem
% eine Experimen-talvariable auf wiederholbare Weise manipuliert unddie Wirkung
% der Manipulation gemessen wird. DerUntersuchungsgegenstand wird entweder in
% seiner na-türlichen Umgebung (im »Feld«) oder in künstlicherUmgebung (im
% »Labor«) untersucht, wodurch wesentlichdie Möglichkeiten der Umgebungskontrolle
% beeinflusstwerden. (Balzert, S. 286)

% ------------------------------------- TEIL Ingenieurlastige Arbeiten ----------------------------------------------------
\section{Projektinformationen}

In diesem Abschnitt wird aufgezeigt welches Vorgehensmodell und welche Methoden verwendet wurden zur Abwicklung dieses Projektes.
Auch wird aufgezeigt wie das Projekt organisiert ist.

\subsection{Vorgehensmodell}

% TODO: vorgehensmodell festsetzen
% SoDa? KanBan? Quellen?


\subsection{Projektteam}

Die folgende Tabelle~\ref{tab:projectmembers} listet alle Personen auf die an diesem Projekt beteiligt sind.

\begin{table}[H]
    \begin{tabular}{l p{3.2cm}}
        \toprule
        \bfseries Person   & \bfseries Rollen \\
        \midrule
        Konrad Bächler     & Auftraggeber \\
        \midrule
        Carolyn Bächler    & Auftraggeber \\
        \midrule
        Arnold Dieter      & Betreuungsperson \\
        \midrule
        tbd.               & Experte \\
        % TODO: find out who's the expert
        \midrule
        Moritz Küttel      & Student \\
        \bottomrule
    \end{tabular}
    \caption{People involved in the project}\label{tab:projectmembers}
\end{table}

\subsection{Quellcode}

Der \LaTeX-Quellcode für diesen Bericht ist auf codeberg.org in diesem Repository zu finden:
\url{https://codeberg.org/mkuettel/ba}

%TODO: add links to all the source code

\subsection{Projektboard und Issue-Tracker}

Die Abbildung~\ref{fig:projectboard} zeigt das Projektboard, welches für Projektmanagement und Controlling verwendet wird.
Jede Karte auf dem Projektboard ist ein Issue aus dem Issue-Tracker:

\url{https://codeberg.org/mkuettel/ba/issues}

Der Issue-Tracker ist zugleich ein Backlog. Issues können nach Bedarf aus dem BackLog entnommen werden und zum Projektboard hinzugefügt werden.

Das Projektboard ist ein KanBan-Board und enthält drei Spalten: ''To Do'', ''In Progress'' und ''Done''.
Diese Issues aus dem Issue-Tracker können dann je nach fortschritt in die Spalten einsortiert werden. 
Ziel von KanBan ist es eines nach dem anderen zu machen und sich auf einen Task zu fokussieren. Man sollte also nie mehr als einen Issue in der Spalte ''In Progress'' haben.
%TODO: definition of done
% https://de.wikipedia.org/wiki/Kanban_(Softwareentwicklung)

%TODO: describe how issues are categorized (labels, milestones, projects etc.)

Das KanBan-Board ist hier zu finden:

\url{https://codeberg.org/mkuettel/BA/projects/125}


\begin{figure*}[ht]
    \includegraphics[width=1.0\textwidth]{project-board.png}
    \caption{CodeBerg Project Board}
    \label{fig:projectboard}
\end{figure*}


Ein einziger Issue sollte nie mehr Aufwand machen als acht Stunden Arbeit. Diese Regel hilft die Issues kleiner zu halten und genauere Aussagen treffen zu können, wo man im Projekt steht.

Wird an einem Issue gearbeitet wird dies im Journal vermerkt \& in der Git-Historie hinterlegt.
Das Arbeitsjournal ist im Anhang~\ref{sec:journal} zu finden.


Die Länge eines Sprints in KanBan ist nicht genau definiert, sondern der Sprint ist fertig wenn alle Arbeit für den jeweiligen Sprint erledigt wurde.

\subsection{Ermittlung offener Projektrahmenbedingungen}
\label{sub:RequirementsEngineering}

Zweimal pro Woche trifft sich das Projektteam wo Fragen und das Weitere
Vorgehen besprochen wird. Es wird jeweils ein Meeting-Protokoll erstellt (siehe
Anhang~\ref{ch:meetingnotes}), worin festgehalten wird was besprochen wurde.

Die Anforderungen entspringen nun aus diesen Diskussionen und Protokollen und sowie aus der Aufgabenstellung.
Auch Vorgaben der Hochschule Luzern wurden als Anforderung aufgenommen.

Fragen bezüglich Anforderungen werden jeweils als Traktanden für das nächste Meeting aufgenommen und dann Besprochen oder direkt mit dem Auftraggeber geklärt.

Für die Liste von Anforderungen siehe Abschnitt~\ref{sec:Anforderungen}.

\subsection{Projektanforderungen / Anforderungsanalyse}
\label{sub:Anforderungen}

\subsection{Einschränkungen und Abgrenzungen}

% TODO: Scope in der Einleitung
% TODO: Scope definieren

\subsection{Systemarchitektur}

\subsection{Komponentendesign}

\subsection{Umsetzung / Programmierung}

\subsection{Testing}


% !TEX root = BA-Bericht.tex
\chapter{Realisierung}
% TODO Beschreibung der Umsetzung der definierten Ziele, einschliesslich der aufgetretenen Schwierigkeiten und Einschränkungen

% Dies ist das Hauptkapitel Ihrer Arbeit! Hier wird die Umsetzung der eigenen Ideen und Konzepte
% (Kapitel 3) anhand der gewählten Methoden (Kapitel 4) beschrieben, inkl. der dabei aufgetretenen
% Schwierigkeiten und Einschränkungen.

\section{Projektmanagement}

\subsection{Meilenstein 1: Grobkonzept}

\subsection{Meilenstein 2: Zwischenpräsentation}

\subsection{Resultate}

Die folgende Tabelle~\fullref{tab:resultate} beschreibt alle Resultate und Unterresultate die während dieser Bachelorarbeit erstellt werden.
Diese Tabelle wurde zusammengestellt anhand der Anforderungen im Abschnitt~\ref{sec:Anforderungen}.

Jedes Resultat hat einen eindeutigen Identifier in  der Spalte ''Nr'' aufgelistet.
Die Spalte ``Anforderung'' bezieht sich darauf, welche Anforderungen für das jeweilige Resultat massgebend sind.
In der letzten Spalte ''Geschätzter Arbeitsaufwand'', wurde zu beginn des Projekts abgeschätzt wie viel Arbeitsaufwand (in Stunden) das Resultat verursacht.

\begin{longtable}{p{0.8cm} l p{3.5cm} p{2cm}}
    \toprule
    \bfseries Nr & \bfseries Resultat & \bfseries Anforderung& \multicolumn{1}{p{3cm}}{\bfseries Geschätzter Arbeitsaufwand} \\
    \midrule \endhead
    D            & \textbf{Dokumentation}                                       & \reqref{DOCS} & \textbf{Total 44h}  \\
    \midrule                                                               
    D.1          & \; Dokumenten Layout                                &       & 8h   \\
    D.2          & \; Aufbau des Berichts                              &       & 4h   \\
    D.3          & \; Titelseite                                       &       & 2h   \\
    D.4          & \; Zusammenfassung / Abstract                       &       & 2h   \\
    D.5          & \; Einleitung                                       &       & 4h   \\
    D.6          & \; Beschreibung Motivation / Problem                &       & 2h   \\
    D.7          & \; Beschreibung der Aufgabenstellung                &       & 2h   \\
    D.8          & \; Beschreibung der Ziele / Vision                  &       & 1h   \\
    D.9          & \; Fragestellungen / Hypothesen                     &       & 2h   \\
    D.11         & \; Reflektion / Fazit                               &       & 3h   \\
    D.12         & \; Persönliches Projektfazit                        &       & 1h   \\
    D.13         & \; Ausblick                                         &       & 4h   \\
    D.14         & \; Anhang                                           &       & 3h   \\
    D.15         & \; Zwischenpräsentation                             & \reqref{PRES}  & 6h  \\
    \midrule                                                               
    R            & \textbf{Research State of the Art}                           & \reqref{SDTF} \reqref{DOCS}  & \textbf{Total 40h} \\
    \midrule                                                               
    R.1          & \; Literatur sammeln (Recherche)                    &       & 12h  \\
    R.2          & \; Bibliographie erstellen                          &       &  4h  \\
    R.3          & \; P2P Networks                                     &       &  2h  \\
    R.3.1        & \;   - Latenz / Bandbreite / Performanz             &       &  2h  \\
    R.4          & \; Beschreibung I2P                                 &       & 10h  \\
    R.4.1        & \; -- Begrifflichkeiten                              &       &  4h  \\
    R.4.2        & \; -- Funktionsweise                                 &       &  2h  \\
    R.4.3        & \; -- Bandbreite                                     &       &  2h  \\
    R.4.4        & \; -- Latenz                                         &       &  2h  \\
    R.5          & \; Deployment von Testnetzwerken                    &       &  4h  \\
    R.6          & \; Beschreibung der Wissenschaftlichen Methode      &       &  2h  \\
    R.7          & \; Metriken für die Auswertung                      & \reqref{TPER} \reqref{TISO} \reqref{TREP}  &  6h  \\
    \midrule                                                               
    K            & \textbf{Testkonzept          }                               & \reqref{TKON} \reqref{DOCS}  & \textbf{Total 46h}  \\
    \midrule                                                               
    K.1          & \; Beschreibung Ideen / Konzepte                    &       &  4h  \\
    K.2          & \; Anforderungen an den Teststand                   &       &  4h  \\
    K.3          & \; Teststrategie                                    &       &  4h  \\
    K.4          & \; Architektur Teststand                            &       &  4h  \\
    K.5          & \; Komponentendiagramm                              &       &  2h  \\
    K.6          & \; Beschreibung was gemessen werden soll            &       &  8h  \\
    K.7.1        & \; -- Bandbreite                                     & \reqref{TLIM} &  2h  \\
    K.7.1        & \; -- Anzahl Tunnels                                 & \reqref{TCNF}    &  2h  \\
    K.7.1        & \; -- Latenz von Nachrichten                         & \reqref{TLAT}    &  2h  \\
    K.7.1        & \; -- Ressourcenauslastung eines Knotens             & \reqref{TPER}    &  2h  \\
    K.8          & \; Beschreibung wie gemessen wird                   &       & 10h  \\
    K.8.1        & \; -- Isolation des Netzwerks                        & \reqref{ORDR}    &  2h  \\
    K.8.1        & \; -- Verschiedene Netzwerksegmente                  & \reqref{ORDR}    &  2h  \\
    K.8.2        & \; -- Latenz                                         & \reqref{ORDR}    &  2h  \\
    K.8.3        & \; -- Bandbreite                                     &       &  2h  \\
    K.8.4        & \; -- Konfigurationsmöglichkeiten                    & \reqref{TCNF} &  2h  \\
    K.9.5        & \; \glsname{ci}                                     & \reqref{TVRS} &  6h  \\
    K.9.6        & \; Beschreibung der Auswertungsmethode              &               &  4h  \\
    \midrule                                                               
    S            & \textbf{Teststand Design und Implementation}                 & \reqref{TINF} \reqref{DOCS} & \textbf{Total 124h} \\
    \midrule
    S.1          & \; Software Design                                  &       &  16h \\
    S.1          & \; Implementation                                   &       &  72h \\
    S.2.1        & \; -- Deployment des Testnetzwerkes                  & \reqref{TVRS} \reqref{TPER} &  8h \\
    S.2.2        & \; -- Netzwerksegmentierung                          & \reqref{TISO} &  8h \\
    S.2.3        & \; -- Konfigurationsmöglichkeiten                    & \reqref{TCNF} &  8h \\
    S.2.4        & \; -- Skalierung                                     & \reqref{TSCL} &  8h \\
    S.2.5        & \; -- Bandbreitenbeschränkung                        & \reqref{TLIM} &  8h \\
    S.2.6        & \; -- Reproduzierbarkeit                             & \reqref{TREP} &  8h \\
    S.2.7        & \; -- Latenzmessung                                  & \reqref{TLAT} &  8h \\
    S.2.8        & \; -- Messung der Ressourcenauslastung               & \reqref{TPER} &  8h \\
    S.2.9        & \; -- Verschiedene Testaufbauten                     & \reqref{TVRS} &  8h \\
    S.3          & \; Test des Labors                                  &       &  24h \\
    S.4          & \; Handbuch für den Teststand                       &       &  12h \\
    S.4.1        & \; -- Installation                                   &       &   2h \\
    S.4.3        & \; -- Konfiguration                                  &       &   2h \\
    S.4.3        & \; -- Ausführen von Messungen                        &       &   2h \\
    S.4.3        & \; -- Beschreibung der gesammelten Messdaten         &       &   4h \\
    S.4.3        & \; -- Beispiele                                      &       &   2h \\
    \midrule                                                                        
    A            & \textbf{Messung und Auswertung}                              & \reqref{EVAL} \reqref{DOCS}   & \textbf{Total 52h}  \\
    \midrule
    A.1          & \; Sammlung an Messdaten für die Auswertung         &        &  12h  \\
    A.2          & \; Beschreibung der Auswertungsmethode              &        &   4h  \\
    A.3          & \; Auswertung der Messungen                         &                    & 20h  \\
    A.3.1        & \; -- Einfluss der Knoten auf die Latenz             & \reqref{TLAT}      &  6h  \\
    A.3.2        & \; -- Einfluss der Anzahl Verbindungen auf die Latenz& \reqref{TLAT} \reqref{TLIM} &  6h  \\
    A.3.3        & \; -- Einfluss der Bandbreite auf Latenz             & \reqref{TLAT} \reqref{TLIM} &  6h  \\
    A.3.4        & \; -- Äussere Einflüsse / Unreinheiten               & \reqref{TREP} \reqref{TISO} &  2h  \\
    A.4          & \; Verschiedene Diagramme/Grafiken                  &       &  8h  \\
    A.5          & \; Auswertung der Anforderungen an den Teststand    & \reqref{TINF} &  4h  \\
    A.6          & \; Zusammenfassung der Auswertung                   &       &  4h  \\
    \midrule                                                               
    P            & \textbf{Project Management Dokumentation}                    & \reqref{DOCS} \reqref{ITER}  &  \textbf{Total 54h}  \\
    \midrule
    P.1          & \; Beschreibung Projektorganisation                 &       &  1h  \\
    P.3          & \; Projektmanagement Methode                        &       &  2h  \\
    P.2          & \; Beschreibung Projektumfang                       &       &  2h  \\
    P.4          & \; Projektplanung                                   &       &  8h  \\
    P.5          & \; Liste von Requirements                           &       &  4h  \\
    P.6          & \; Liste von Resultaten                             &       &  4h  \\
    P.9          & \; Arbeitsjournal                                   &       &  4h  \\
    P.10         & \; Meeting-Protokolle und Notizen                   &       & 29h  \\
    \midrule                                                               
                 & \bfseries  Geschätzter Arbeitsaufwand               & \textbf{Total} & \bfseries 360h \\
    \midrule
    \bottomrule
    \caption{Resultate}
    \label{tab:resultate}
\end{longtable}

Diese Liste von Resultaten ist auch ein guter Ausgangspunkt um davon Issues für den Issue-Tracker zu erstellen.
Jeder issue kann nun auch sortiert werden nach Resultat-Kategorien z.B. ''Recherche'', ''Projektmanagement'', ''Dokumentation'', ''Evaluation'', ''Präsentation'', etc.

\newpage

\section{Systemarchitektur}

Grob besteht die entwickelte Software für den Teststand aus drei teilen.

\begin{enumerate}
    \item \textbf{Tester-Software}: Hier ist die Deployment-Konfiguration für die Test-VM zu finden, sowie die verwendeten Tools zur Auswertung.
    \item \textbf{Test-VM:} Hier ist die Konfiguration für das 
    \item \textbf{Testnetzwerk:} Das Testnetzwerk besteht aus i2pd-Containern sowie aus dem Reseed-Container der verwendet wird zum Bootstrapping des Testnetzwerks.
\end{enumerate}

Die Abbildung~\fullref{fig:architektur-diagramm} zeigt in einem Komponenten-Diagram die Systemarchitektur auf.


\begin{landscape}% Landscape page
\begin{figure*}[ht]
  \includegraphics[height=0.85\textwidth]{include/uml/componentDiagram.png}
  \caption{Architektur-Diagramm}\label{fig:architektur-diagramm}
\end{figure*}
\end{landscape}% Landscape page


\section{Komponentendesign}

\subsection{Deployment der Test-VM}

NixOps erlaubt es eine NixOS-Systemkonfiguration auf diverse Arten von Maschinen zu deployen.

\subsection{Konfiguration}

Die Konfiguration für

\subsection{Recompose-Skript}

Das Recompose-Skript ist dafür verantwortlich das Docker-Netzwerk aufzubauen und zu managen.

Als wichtigster Punkt wird hier der 



\subsection{Deployment von der Container}

Da ich bereits NixOps zum deployment der Container verwendet habe, schien es naheliegend die NixOS-Container Technologie zu verwenden.

Jedoch bin ich mit diesem Ansatz an mehreren Stellen an Grenzen gestossen.

Als erstes traten Probleme mit der Netzwerkkonfiguration. Die NixOS-Container implementation scheinte die 

• Erst Probleme mit der Netzwerkkonfiguration
• Probleme mit Docker-Kompatibilität
• Skaliert nicht auf mehr als ein paar dutzend Container. Zu viel
RAM benötigt lediglich zum Berechnen der verschiedenen
Container-Konfigurationen.

Es hat 

Anfangs wurde auf ein Ansatz mit 


\subsection{Skalierung der i2pd-Container}

Der erste Lösungsansatz mit NixOS-Containern hat sich als ungeeignet herausgestellt. \seereq{TSCL}

Dieser Ansatz wurde verworfen, weil dies dazu geführt hat, das die i2pd-Container untereinander nicht kommunizieren konnten.

\subsection{Bootstrapping}

Die I2P-Router im öffentlichem I2P-Netzwerk können sich hier einfach an den öffentlich verfügbaren Reseeder eine \lstinline|su3|-Datei herunterladen.
Diese \lstinline|su3|-Datei beinhaltet RouterInfos. 
Diese RouterInfos beinhalten die nötigen Informationen wie public-key und Netzwerkadresse, um die Kommunikation zu starten.

Im privaten und abgeschotteten Testnetzwerk sind die öffentlichen Reseed-Server jedoch nicht erreichbar.
Deshalb muss ein anderer Weg gefunden werden das private I2P-Netzwerk zu bootstrappen.



Denn der Floodfill-Anstatz entspricht nicht unbedingt der Realität, denn die Reseeder-Knoten können liefern nicht allen I2P-Routern dieselbe Liste an Peers.

Auch kann die Konfiguration des Reseeders eine wichtige Rolle spielen, denn dieser bestimmt mit der Liste an Peers die er den I2P-Routern liefert den Konnektivitätsgrad der Knoten.


\subsection{Konfiguration}

Die Konfiguration des Teststands kann mittels einer json-Konfigurationsdatei einfach angepasst werden.
Diese kann, falls nötig einfach von verschiedenen Orten gelesen werden.

\subsection{Resultate zurücklesen}

Just read logs? Use some Docker cp magic?

\section{Umsetzung / Programmierung}

Der Quellcode für das Testnetzwerk ist im folgenden Repository zu finden:

\subsection{VM / Container Netzwerk}

NixOS erlaubt es deklarativ Container zu definieren.

% \begin{lstlisting*}[language=nix,label=src:nixos-container,caption=Beispiel wie Nixos Container definiert werden können]
%
%     boot.enableContainers = true;
%     containers = {
%         "container1" = {
%             config = { config, pkgs, ... }:
%                 {
%                     # NixOS system configuration for this container
%                     services.httpd.enable = true;
%                 }
%             # more container settings ... 
%         };
%         # "container2" .... 
%     }
% \end{lstlisting*}

Wichtig dabei ist, dass es sich hierbei nicht um einen Docker-Container, sondern um NixOS-Container handelt.
Im Hintergrund brauchen jedoch beide Container-Arten die gleichen Linux-Kernel-Features (namespaces, cgroups, ... ).


Siehe \lstinline|containers.nix| in der \lstinline|i2p-testnet| Repository für die Effektive implementation


\url{https://codeberg.org/mkuettel/i2p-testnet}.

\subsection{Bootstrapping}

\subsubsection{Reseed-Server}

Dieser Ansatz hat den Vorteil, das so das Testnetzwerk schnell aufgebaut werden kann.

Die \lstinline|i2pd|-Implementation beinhaltet selber keinen 

Die Go-Implementation dieses Reseed-Servers ist hier zu finden:

\url{https://codeberg.org/diva.exchange/i2p-tools}

Diese version basiert auf der folgenden GitHub version:

\url{https://github.com/MDrollette/i2p-tools}

\subsubsection{Ablauf}


Der Bootstrapping-Vorgang im Testnetzwerk wird wie folgt abgehandelt:

\begin{enumerate}
    \item Erstellen der Container für den Reseed-Sever und den I2P-Knoten
    \item Der Reseed-Server erstellt die nötigen Schlüssel und Zertifikate, um einerseits später den HTTPS-Reseed-Server zu starten und andererseits die su3-Dateien zu signieren.
    \item Der Reseed-Server-Container fängt an zu warten bis er alle RouterInfo's von allen I2P-Knoten erhalten hat
    \item Die I2P-Router werden kurzzeitig gestartet, bis diese Ihre RouterInfo Datei erstellen. Die I2P-Router werden anschliessend wieder gestoppt, da der Reseed-Server im Moment noch nicht in Betrieb ist, da ihm die RouterInfos noch fehlen.
    \item Die I2P-Knoten-Container warten anschliessend bis der HTTPS-Reseed-Server verfügbar ist.
    \item Auf der Test-VM kopiert ein Hintergrund-Job alle RouterInfos vom I2P-Container in die NetDb des Reseed-Server-Containers.
    \item Nach kurzer Zeit hat der Reseed-Server-Container alle benötigten RouterInfos. Nun generiert dieser die su3-Datei aus den RouterInfos und startet anschliessend nun effektiv den HTTPS-Reseed-Server.
    \item Die I2P-Knoten-Container können nun den HTTPS-Reseed-Server erreichen und somit wird der I2P-Router gestartet.
    \item Die I2P-Router laden nun die su3-Datei herunter und können so die anderen Knoten ausfindig machen und Ihre Arbeit starten.
\end{enumerate}

Welche Reseeder ein I2P-Router anfragt, kann mittels der \lstinline|i2pd|-Konfigurationsoption \lstinline|urls| festgelegt werden.


\clearpage
\begin{landscape}% Landscape page
\begin{figure*}[ht]
  \includegraphics[height=0.85\textwidth]{bootstrap-diagram.png}
  \caption{Der Boostrapping Prozess}\label{fig:bootstrap-diagram}
\end{figure*}
\end{landscape}% Landscape page


\subsection{Reproduzierbarkeit}

* Nix
* Pinning
    
\cite{noauthor_nixops_nodate-5}


\subsection{Isolieren}

Während unserer Tests wollen wir nur Traffic unserer eigenen Knoten im Netzwerk 

docker-compose respektive docker hat die option ein eigenes Netzwerk zu erstellen:

\subsubsection{Abgrenzen vom Haupt-Netzwerk}

Die Konfigurationsoption \lstinline|netid| erlaubt es die Netzwerknummer zu definieren.
Beim realen I2P-Netzwerk ist diese standardmässig auf \lstinline|2| gesetzt.

VM / Container, Konfigurationsoption \lstinline|nat|

Nix für container scheint nicht so gut zu funktionieren

Weil:

Das Builder der systemkonfigurationen inkl. der container getestet mit 100 konnte nicht durchlaufen da nix viel memory braucht (ich habe nach 16GB und 20min warten aufgehört da mein Laptop am swappen war)
Fehlerhafte netzwerkkonfiguration by default, falsche routen und ip konfigurationen (fehlende definition der netzwerkmaske)
Container selber sind unnötig gross

\section{Testing}

CI? most don't allow this...

have a machine to deploy to

\section{Benutzerhandbuch}




\chapter{Evaluation und Validation}
\label{ch:Eval}

% Auswertung und Interpretation der Ergebnisse. Nachweis, dass die Ziele erreicht wurden, oder
% warum welche nicht erreicht wurden.

Quantitative Auswertung

\section{Validation}

Damit weitere 


\section{Vergleich mit Anforderungen}
\label{sec:VergleichAnforderungen}
% TODO Vergleich mit Anforderungen Soll<->Ist

\seereq{}



\section{Technische Aspekte}
% TODO Evaluation der verwendeten Hilfsmittel

\section{Vorgehen}
% TODO Evaluation der verwendeten Arbeitsprozesse


% !TEX root = BA-Bericht.tex
\chapter{Ausblick}
\label{ch:Ausblick}
$% umbenennen Abschluss

% Reflexion der eigenen Arbeit, ungelöste Probleme, weitere Ideen.


\section{Projekt Fazit}
% TODO Das Fazit des Projektes, auch eine Unterteilung in \subsection mit persönlichem und Projektfazit ist möglich

Diese Arbeit konnte mit dem erstellen eines I2P-Testnetzwerks nun einen Grundstein legen für weitere Untersuchungen.


\section{Persönliches Fazit}

Ich finde dies und das....

\section{Ausblick}
% TODO Was ist für Folgeprojekte wichtig, welche Lehren können gezogen werden, was sind noch offene Fragen?

Diese Arbeit 

\begin{itemize}
    \item Es gibt andere Möglichkeiten Nachrichten über das I2P-Netzwerk zu versenden.
    Unter anderem gibt es das SAM-Protokoll.  Es könnte möglich sein bessere Performance mit diesem Protokoll zu erreichen anstatt einen SOCKS5 oder HTTP-Proxy zu verwenden. \cite{noauthor_sam_nodate}
    Der Nachteil bestände jedoch darin, dass die darüberliegende Datenhaltungsschicht dieses Protokoll implementieren müsste und fest an I2P gekoppelt wäre.
    \item Horizontal Skalieren
    \item Cloves / Garlic-Routing
    \item Peer-Profiling
\end{itemize}


\newpage

\pagenumbering{Roman}

\appendix
% Verhindert das Einfügen von Kapiteltitel kleiner als \chapter
\addtocontents{toc}{\protect\setcounter{tocdepth}{0}}

\glsaddall
\printglossary

\listoffigures

\listoftables

% \huge{Formelverzeichnis}
\listofmyequations \pagebreak


\printbibliography

% TODO Anhänge anfügen
% Wir haben dies jeweils über \chapter gelöst
% \includepdf[pages=-]{PDF-ANHANG}


% TODO Anhänge anfügen
% Wir haben dies jeweils über \chapter gelöst
% \includepdf[pages=-]{PDF-ANHANG}


% TODO: aufgabenstellung einfügen
% \includepdf[
%     addtotoc={1,chapter,0,Assignment / Aufgabenstellung,ch:assignment},
%     pages=-
% ]{include/assignment.pdf}


\chapter{Meeting Notes}
\label{ch:meetingnotes}


\includepdf[
    addtotoc={1,section,1,Notes -Kick-Off-Meeting 24.02.2021,sec:meeting_24_02_2021},
    pages=-
]{Meetings/24-02-2021.pdf}

\includepdf[
    addtotoc={1,section,1,Meeting Notes 26.02.2021,sec:meeting_26_02_2021},
    pages=-
]{Meetings/26-02-2021.pdf}


% \chapter{Other Markdown stuff?}
% \label{ch:OtherMarkdownStuff}
% \markdownInput[shiftHeadings=1]{include/somwhere/something.md}



\end{document}
